\documentclass[12pt]{article}

% UTF-8 encoding and fonts
\usepackage[utf8]{inputenc}
\usepackage[T1]{fontenc}
\usepackage{lmodern}

% Page setup
\usepackage[margin=1in]{geometry}
\usepackage{setspace}
\onehalfspacing

% Typography
\usepackage{microtype}

% Math and symbols
\usepackage{amsmath,amssymb}

% Graphics
\usepackage{graphicx}
\usepackage{float}
\usepackage{subcaption}

% Tables
\usepackage{booktabs}
\usepackage{array}
\usepackage{multirow}
\usepackage{threeparttable}
\usepackage{longtable}
\usepackage{pdflscape}
\usepackage{siunitx}
\sisetup{detect-all=true, group-separator={,}, group-minimum-digits=4}

% Bibliography
\usepackage{natbib}
\bibliographystyle{aer}

% Hyperlinks
\usepackage{hyperref}
\hypersetup{
    colorlinks=true,
    linkcolor=blue,
    citecolor=blue,
    urlcolor=blue
}
\usepackage[nameinlink,noabbrev]{cleveref}

% Timing data
\IfFileExists{timing_data.tex}{\newcommand{\apepcurrenttime}{3m}
\newcommand{\apepcumulativetime}{3m}
}{
  \newcommand{\apepcurrenttime}{N/A}
  \newcommand{\apepcumulativetime}{N/A}
}

% Captions
\usepackage{caption}
\captionsetup{font=small,labelfont=bf}

% Section formatting
\usepackage{titlesec}
\titleformat{\section}{\large\bfseries}{\thesection.}{0.5em}{}
\titleformat{\subsection}{\normalsize\bfseries}{\thesubsection}{0.5em}{}

% Float notes (for figure footnotes)
\newcommand{\floatfoot}[1]{\par\vspace{0.5em}\footnotesize #1}

% Custom commands
\newcommand{\E}{\mathbb{E}}
\newcommand{\Var}{\text{Var}}
\newcommand{\Cov}{\text{Cov}}
\newcommand{\ind}{\mathbb{I}}
\newcommand{\sym}[1]{\ifmmode^{#1}\else\(^{#1}\)\fi}

\title{Brewing Dynasties and Broken Ladders: State Prohibition and the\\ Destruction of the German-American Economic Elite, 1870--1920}
\author{APEP Autonomous Research\thanks{Autonomous Policy Evaluation Project. Correspondence: scl@econ.uzh.ch} \and @ai1scl}
\date{\today}

\begin{document}

\maketitle

\begin{abstract}
\noindent
Between 1881 and 1919, thirty-three U.S.\ states adopted prohibition, destroying the German-dominated brewing industry. Using a state-year panel of 47 states across six census years (1870--1920) constructed from published Census tabulations, I estimate prohibition's effect on the German-born population share. Average two-way fixed-effects estimates are positive, reflecting selection: prohibition states had small German populations. However, interacting treatment with pre-prohibition brewing intensity and German enclave status reveals that prohibition \textit{reduced} German-born shares precisely where Germans were concentrated. In states with above-median German populations, the net effect of prohibition was a 2.1 percentage point reduction in the German-born share ($p < 0.01$). In high-brewing-intensity states, the net effect is similarly negative. These heterogeneous effects are consistent with the destruction of an ethnic enclave economy driving German out-migration.
\end{abstract}

\vspace{1em}
\noindent\textbf{JEL Codes:} J15, J61, N31, N32, L66 \\
\noindent\textbf{Keywords:} prohibition, German-Americans, ethnic enclaves, brewing industry, difference-in-differences, migration

\newpage

%% ===================================================================
\section{Introduction}
%% ===================================================================

On the morning of January 17, 1920, the Eighteenth Amendment took effect and the United States became a dry nation. But for German-American brewers in Kansas, Georgia, and two dozen other states, the end had come years earlier. When Kansas adopted statewide prohibition in 1881, it shuttered some of the most profitable businesses in the state---nearly all of them German-owned. By 1919, thirty-three states had followed suit, dismantling a \$800-million industry that employed 88,000 workers directly and 300,000 in dependent sectors \citep{okrent2010, warburton1932}. The brewing industry was not merely a business: it was the economic backbone of German-America, the engine that had propelled an immigrant community from steerage passengers to an economic elite in a single generation.

This paper asks what happens to an immigrant community when the state systematically destroys its principal ethnic enclave economy. Specifically, I study the effects of state prohibition laws on the German-born population share across U.S.\ states, exploiting the staggered adoption of statewide prohibition across 33 states between 1881 and 1919. I construct a state-year panel of 47 states across six decennial census years (1870, 1880, 1890, 1900, 1910, and 1920) using aggregate population tabulations published by the U.S.\ Census Bureau \citep{gibsonjung2006}. This allows me to track how the geographic distribution of German-born immigrants shifted in response to prohibition---whether Germans left states that destroyed their primary industry, or whether the ethnic enclave economy proved resilient to regulatory destruction.

The main empirical challenge is that prohibition adoption was not random. States that went dry tended to be rural, Southern, and Protestant---precisely the states where German settlement was sparse. A naive comparison of German-born population shares in dry versus wet states therefore conflates the treatment effect with selection: prohibition states had \textit{fewer} Germans to begin with. My baseline two-way fixed-effects (TWFE) specification confirms this selection pattern: the average treatment coefficient on the German-born share is positive (+0.015, $p < 0.01$), seemingly suggesting that prohibition \textit{increased} the German-born share. But this estimate is misleading. It reflects the fact that the pool of treated states is dominated by Southern states with very low German populations, where the positive coefficient captures differential secular trends rather than a causal effect of prohibition.

The paper's main contribution is to move beyond the misleading average effect and identify the \textit{heterogeneous} effects of prohibition. I interact the treatment indicator with two pre-prohibition characteristics that capture where prohibition's industry-destruction channel should bind most tightly: (i) \textit{brewing intensity}, measured as the number of breweries per 100,000 population from the 1870 Census of Manufactures \citep{cmfdata2025}, and (ii) \textit{German enclave status}, an indicator for states with above-median German-born population shares in 1890 \citep{censusbulletin357}. In states where these characteristics are high---where the German-American brewing economy was large and vibrant---prohibition should destroy more ethnic economic infrastructure and generate stronger incentives for German out-migration.

The heterogeneous effects tell a dramatically different story from the average. When I interact treatment with a high-brewing-intensity indicator, the interaction coefficient is $-0.027$ ($p < 0.01$), while the main treatment effect is $+0.019$ ($p < 0.01$). The net effect in high-brewing states is therefore $+0.019 - 0.027 = -0.009$: prohibition \textit{reduced} the German-born share in states with substantial brewing industries. The German enclave interaction is even more striking: the coefficient on treatment $\times$ German enclave is $-0.043$ ($p < 0.01$), implying that prohibition reduced the German-born share by 2.1 percentage points in states with above-median German populations. These results are consistent with the hypothesis that prohibition drove German out-migration by destroying the ethnic enclave economy---but only where that economy was large enough to matter.

The results are robust across specifications. Alternative control groups---dropping border states, restricting to Southern versus Western states, dropping the District of Columbia---yield qualitatively identical patterns. A dose-response specification using the duration of prohibition exposure shows a positive (though imprecisely estimated) relationship between years under prohibition and the German-born share in levels, but a significant positive relationship for log German-born population, consistent with cumulative effects. Randomization inference for the baseline TWFE yields a $p$-value of 0.29, correctly indicating that the positive average coefficient is not informative about the causal mechanism, and underscoring the importance of the heterogeneity analysis that reveals where the effect actually operates.

This paper contributes to several literatures. First, I add to the growing body of work on immigrant settlement and mobility during the Age of Mass Migration \citep{hattonwilliamson1998, abramitzky2012, abramitzky2014, tabellini2020}. While most of this literature studies factors that facilitate or impede assimilation, I study the consequences of forcibly dismantling the economic infrastructure of an established immigrant community. The results suggest that ethnic enclave economies are not merely transitional institutions \citep{portesmanning1986, edin2003, burchardi2019}: their destruction reshapes the geographic distribution of immigrant populations.

Second, I contribute to the economics of prohibition. The existing literature focuses primarily on crime \citep{owens2011, dills2005}, alcohol consumption \citep{miron1999}, and public health \citep{fisher1928}. By studying the distributional consequences of prohibition for the ethnic group most concentrated in the targeted industry, I provide a new perspective on one of America's most studied policy experiments. The finding that prohibition's demographic effects operated heterogeneously---concentrated in states where the German brewing economy was substantial---highlights how ostensibly neutral regulatory policies can have sharply differential impacts on immigrant communities.

Third, the paper speaks to recent work on the economic consequences of ethnic hostility and forced cultural assimilation \citep{fouka2019, moser2014}. The World War I period saw both prohibition and anti-German persecution simultaneously. While I cannot fully disentangle these channels with aggregate state-level data, the heterogeneous pattern---effects concentrated where brewing was important rather than where anti-German sentiment was strongest---is more consistent with an economic mechanism (industry destruction) than a cultural one (nativist pressure).

Fourth, I contribute methodologically by demonstrating how modern staggered difference-in-differences methods---including the \citet{sunandabraham2021} interaction-weighted estimator, the \citet{callaway2021} group-time estimator, and the \citet{goodmanbacon2021} decomposition---can rescue an otherwise uninformative average effect. The positive baseline TWFE coefficient is a textbook example of how selection into treatment can generate misleading estimates. The interaction specifications reveal a rich pattern of heterogeneity that the average completely obscures, illustrating why modern difference-in-differences methods that accommodate effect heterogeneity are essential for policy evaluation in historical settings.


%% ===================================================================
\section{Institutional Background}\label{sec:background}
%% ===================================================================

\subsection{The German-American Brewing Industry}

German immigrants transformed the American brewing industry in the second half of the nineteenth century. Before the Civil War, most American beer was English-style ale, brewed in small batches by local producers. The wave of German immigration that began in the 1840s brought not only brewmasters trained in the Bavarian tradition of lager beer but also a consumer base with deep cultural attachment to beer \citep{kolodrubetz1968}. By 1880, German-born immigrants and their children operated approximately 80 percent of all breweries in the United States \citep{downard1973}.

The industry grew rapidly during the Gilded Age. The number of breweries peaked at approximately 4,131 in 1873, then consolidated into fewer but much larger operations \citep{blocker2003}. By 1914, there were approximately 1,400 breweries producing over 66 million barrels annually, with total capital of approximately \$800 million---equivalent to roughly \$18 billion in 2010 dollars. The industry directly employed 88,000 workers and supported an estimated 300,000 in dependent sectors including cooperage, hop farming, malt production, and distribution \citep{okrent2010, warburton1932}.

The brewing elite occupied a distinctive position in American society. Families like the Busches of St.\ Louis, the Pabsts of Milwaukee, and the Ruperts of New York were among the wealthiest in their cities. They built palatial homes, endowed cultural institutions, and wielded significant political influence through the United States Brewers' Association, one of the most powerful trade organizations in the country \citep{kerr1985}. At the same time, the industry provided economic opportunity at every level: from unskilled laborers in the brewhouses to skilled brewmasters, from teamsters delivering barrels to saloonkeepers serving customers. For German-American communities, the brewing industry was not merely a source of employment but a complete economic ecosystem that supported upward mobility from working class to middle class to elite.

The geographic distribution of brewing mirrored the geography of German settlement. States with large German-born populations---Wisconsin, Missouri, Ohio, Pennsylvania, New York, Illinois---were home to the vast majority of the nation's brewing capacity. The 1870 Census of Manufactures, which I use to construct the brewing intensity measure, reveals enormous cross-state variation: some states had more than 20 breweries per 100,000 population, while many Southern and Western states had none at all. This geographic concentration is central to the identification strategy: prohibition should have the largest demographic effects precisely where the German-brewing nexus was strongest.

\subsection{The Staggered Adoption of State Prohibition}

The prohibition movement in the United States evolved over more than half a century before culminating in the Eighteenth Amendment. While the eventual national ban is well known, the state-level prohibition movement created the policy variation I exploit for identification: between 1851 and 1919, thirty-three states adopted statewide prohibition laws at different times, while the remaining fifteen states and the District of Columbia remained wet until the federal amendment took effect on January 17, 1920.\footnote{Including DC and Kentucky (whose state law took effect in 1920), 17 jurisdictions are functionally never-treated in the analysis. After excluding Maine and Kansas as very early adopters, 30 states serve as treated units.}

The first wave of state prohibition began in the 1850s, led by Maine's ``Maine Law'' of 1851. Several other states followed in the 1850s, but most of these early laws were repealed within a decade. Kansas became the first state to permanently enshrine prohibition in its constitution in 1881, and North Dakota entered the Union as a dry state in 1889.

The second and larger wave came in the Progressive Era. Oklahoma adopted prohibition upon statehood in 1907. A cluster of Southern states followed: Georgia and Mississippi in 1908, North Carolina in 1908, Alabama and Tennessee in 1909. A pause followed until 1914--1916, when a dozen states adopted prohibition: West Virginia, Virginia, Colorado, Oregon, and Washington in 1914--1916; Arizona, Arkansas, Idaho, Iowa, and South Carolina in 1915--1916. The final wave came as the Eighteenth Amendment approached ratification: Indiana, New Hampshire, Nebraska, South Dakota, and Utah in 1917; Michigan, Montana, Florida, New Mexico, Wyoming, Nevada, Ohio, and Texas in 1918--1919 \citep{blocker2003}.

Figure~\ref{fig:timeline} shows the distribution of prohibition adoption years across the 33 treated states. The clustering of adoptions in 1916--1919 is evident, with a secondary concentration in 1907--1909. This variation in timing---combined with the 17 never-treated jurisdictions in the analysis sample---provides the staggered treatment variation underlying the difference-in-differences design.

\begin{figure}[H]
    \centering
    \includegraphics[width=0.85\textwidth]{figures/fig1_adoption_timeline.pdf}
    \caption{Distribution of State Prohibition Adoption Years}
    \label{fig:timeline}
    \floatfoot{\textit{Notes:} Histogram showing the year of statewide prohibition adoption for each of the 33 states that went dry before the Eighteenth Amendment (effective January 17, 1920). Sources: \citet{blocker2003}; 1922 \textit{Encyclopaedia Britannica}.}
\end{figure}

The political economy of prohibition adoption varied by region. In the South, prohibition was driven in part by racial politics: white elites sought to control Black access to alcohol, and the temperance movement drew on the same populist energy that produced Jim Crow \citep{okrent2010}. In the Midwest and West, progressive reformers combined moral arguments about alcohol with practical concerns about saloon-based political machines. In both cases, the temperance movement portrayed the German-dominated brewing industry as a foreign threat to American values---a framing that intensified dramatically during World War I \citep{kerr1985}.

But while the legal tide was rising, a cultural storm was gathering. Crucially for identification, the political motivations for prohibition adoption are correlated with the pre-existing German-born share. Southern states that went dry earliest had the smallest German populations, while Midwestern and Great Lakes states with large German communities generally remained wet until the very end (Ohio in 1919) or until the Eighteenth Amendment forced their hand (Wisconsin, Illinois, New York). This negative correlation between German settlement and prohibition timing generates the positive selection bias that I document in the baseline TWFE results.

\subsection{World War I and Anti-German Sentiment}

The timing of state prohibition adoption overlaps substantially with the rise of anti-German hostility during World War I (1914--1918). This overlap creates a central identification challenge: between 1916 and 1919, approximately twenty states adopted prohibition, and the same period saw German-language schools closed, German books burned, and German-Americans subjected to violence and social ostracism \citep{fouka2019}.

The temperance movement explicitly weaponized anti-German sentiment. The Anti-Saloon League published propaganda linking beer to the Kaiser, and prohibitionists argued that the ``German-American Alliance'' was a front for the Kaiser's intelligence services \citep{kerr1985, okrent2010}. In this environment, prohibition was both an industry regulation and an act of ethnic targeting.

The overlap between prohibition and anti-German sentiment means that any observed decline in the German-born population share in prohibition states could reflect either (a) economic displacement from the destruction of the brewing industry or (b) ethnic hostility driving Germans to relocate. My heterogeneity analysis provides indirect leverage on this distinction: if the demographic effect is concentrated in high-brewing states rather than uniformly distributed across all prohibition states, the industry-destruction channel is more likely to dominate. As I show in \Cref{sec:results}, this is precisely the pattern the data reveal.

\subsection{German-American Migration and Settlement Patterns}

The German-born population of the United States followed a distinctive arc during the period under study. Immigration from Germany surged in the 1840s--1880s, peaked in absolute numbers around 1882, and declined sharply after 1893. The share of German-born residents in the U.S.\ population fell from approximately 4.2 percent in 1870 to 1.1 percent in 1920, driven by declining immigration inflows and the aging and mortality of earlier cohorts \citep{gibsonjung2006}.

This national decline provides important context for the state-level analysis. All states experienced a falling German-born share over this period; the question is whether prohibition states experienced a \textit{differentially} larger decline. The TWFE specification with state and year fixed effects absorbs both state-specific levels and common national trends, isolating the within-state deviation from the national trajectory that can be attributed to prohibition.

German settlement patterns were strongly path-dependent. Once a critical mass of German immigrants settled in a state, chain migration, ethnic institutions, and economic networks (including the brewing industry) attracted subsequent waves. States that had attracted early German settlers---Wisconsin, Ohio, Missouri, Pennsylvania, New York---maintained large German-born communities throughout the period. These were precisely the states where the brewing industry was most developed, creating the geographic coincidence between ethnic enclave and industry that makes prohibition's heterogeneous effects interpretable.


%% ===================================================================
\section{Data}\label{sec:data}
%% ===================================================================

\subsection{Aggregate Census Population Data}

The primary dataset is a state-year panel of 47 states across six decennial census years (1870, 1880, 1890, 1900, 1910, 1920). I construct this panel from published aggregate tabulations of the U.S.\ Census of Population, drawing on several complementary sources.

\textbf{Total population and foreign-born population by state.} State-level total population and foreign-born population counts for all census years 1870--1920 come from \citet{gibsonjung2006}, a Census Bureau Working Paper that compiles historical population data from published census volumes. Specifically, I use their Table 14, which reports the total population and the foreign-born population for each state and territory at every decennial census from 1850 to 2000. The foreign-born share is computed as the ratio of these two quantities.

\textbf{German-born population by state, 1890 and 1920.} Direct state-level tabulations of the German-born population are available for two benchmark years. For 1890, I use Census Bulletin No.\ 357 \citep{censusbulletin357}, which tabulates the foreign-born population by country of birth for each state. For 1920, I use Table 6 of Chapter VI in Volume II of the Fourteenth Census of the United States \citep{census1920v2}, which reports the foreign-born white population by country of birth and state.

\textbf{German-born population by state, interpolation for other years.} For the remaining four census years (1870, 1880, 1900, 1910), state-level German-born counts are not readily available in published tabulations. I impute them using a proportional allocation method. Specifically, I take the national total of German-born residents from \citet{gibsonjung2006} Table 4, which reports total foreign-born by country of origin at each census, and allocate it to states in proportion to each state's share of the total German-born population in the nearest benchmark year (1890 for years before 1890; interpolated shares between 1890 and 1920 for years between these benchmarks). This method assumes that the \textit{cross-state distribution} of German-born residents evolved smoothly between the 1890 and 1920 benchmarks, an assumption I discuss in \Cref{sec:robustness}.

\textbf{Sample.} The panel includes 47 states: the 45 states admitted by 1900 (excluding Maine and Kansas, whose early prohibition adoptions in 1851 and 1881 predate the analysis window), plus three territories that gained statehood during the period---Oklahoma (1907), New Mexico (1912), and Arizona (1912)---and the District of Columbia.\footnote{Alaska, Hawaii, and other non-contiguous territories are excluded. Maine and Kansas are excluded because they adopted prohibition decades before the next-earliest state, making them unsuitable as either treated or control units. North Dakota and Oklahoma are missing from the 1870 and 1880 census years, yielding an unbalanced panel.} The resulting panel has 278 state-year observations across the main specifications.

\subsection{Brewing Industry Data}

Data on brewery counts by state come from the Census of Manufactures microdata compiled by \citet{cmfdata2025} and made available through \url{cmfdata.org}. I use the 1870 Census of Manufactures to count the number of brewery establishments in each state and construct a ``brewing intensity'' measure: breweries per 100,000 population. I use the 1870 cross-section rather than later years to ensure that the brewing intensity measure is predetermined relative to most prohibition adoptions, which begin in earnest in 1907.

The 1870 brewing intensity measure varies enormously across states: from zero in several Southern and Western states to over 20 per 100,000 in states like Wisconsin, California, and Pennsylvania. Figure~\ref{fig:map} maps this variation, showing the strong geographic concentration of brewing in the Upper Midwest and Mid-Atlantic states. This variation is strongly correlated with the German-born share (correlation $> 0.6$), reflecting the ethnic character of the industry. I also construct a binary ``high-brewing'' indicator equal to one for states with above-median brewing intensity in 1870.

\begin{figure}[H]
    \centering
    \includegraphics[width=0.85\textwidth]{figures/fig8_map_brewing.pdf}
    \caption{Brewing Intensity by State, 1870}
    \label{fig:map}
    \floatfoot{\textit{Notes:} Breweries per 100,000 population from the 1870 Census of Manufactures \citep{cmfdata2025}. Darker shading indicates higher brewing intensity.}
\end{figure}

\subsection{State Prohibition Adoption Dates}

I compile a comprehensive dataset of state prohibition adoption dates from \citet{blocker2003} and the 1922 \textit{Encyclopaedia Britannica} article on prohibition. For each state, I record the year in which statewide prohibition took effect. The treatment indicator $\text{Treated}_{st}$ equals one if state $s$ had statewide prohibition in effect at the time of census year $t$'s enumeration. States that were never dry before the Eighteenth Amendment are coded as untreated in all years; all states are coded as treated in 1920, when national prohibition was in effect.

\subsection{Summary Statistics}

Table~\ref{tab:summary} presents summary statistics by census year. Several patterns are notable. The mean state population grows from 835,000 in 1870 to 2.2 million in 1920, reflecting both population growth and immigration. The foreign-born share declines modestly from 18.0 percent in 1870 to 12.0 percent in 1920, while the German-born share declines more sharply from 7.4 percent to 1.3 percent, reflecting the cutoff of German immigration after World War I. The treated share rises from 0 percent in 1870--1880 (when only Kansas and Maine were dry, and Kansas adopted in 1881) to 100 percent in 1920 under national prohibition.

\begin{table}[H]
\centering
\caption{Summary Statistics by Census Year}
\label{tab:summary}
\centering
\begin{tabular}[t]{crrrrr}
\toprule
\multicolumn{1}{c}{ } & \multicolumn{5}{c}{Panel A: Full Sample} \\
\cmidrule(l{3pt}r{3pt}){2-6}
Year & N & Mean pop. (000s) & Foreign-born (\%) & German-born (\%) & Treated (\%)\\
\midrule
1870 & 45 & 835 & 18.0 & 7.37 & 0\\
1880 & 45 & 1078 & 16.1 & 3.53 & 0\\
1890 & 47 & 1288 & 16.2 & 3.12 & 2\\
1900 & 47 & 1571 & 13.9 & 2.62 & 2\\
1910 & 47 & 1905 & 14.0 & 1.90 & 15\\
\addlinespace
1920 & 47 & 2195 & 12.0 & 1.25 & 100\\
\bottomrule
\end{tabular}
\end{table}

Table~\ref{tab:balance} presents pre-treatment balance between eventual prohibition states and never-dry control states, measured at the 1890 census. The two groups differ in ways that are consistent with the selection pattern described above: eventual prohibition states are smaller (mean population 968,000 vs.\ 1.85 million), have lower foreign-born shares (14.6\% vs.\ 19.0\%), and have lower German-born shares (2.3\% vs.\ 4.6\%). Interestingly, eventual prohibition states have \textit{higher} brewing intensity (10.5 vs.\ 5.6 breweries per 100,000), suggesting that some states adopted prohibition precisely because the brewing industry was visible and politically salient. These pre-treatment differences underscore the importance of state fixed effects and the heterogeneity analysis: the TWFE specification absorbs time-invariant differences between states, while the interactions isolate the effect of prohibition conditional on the pre-treatment German and brewing presence.

\begin{table}[H]
\centering
\caption{Pre-Treatment Balance (1890)}
\label{tab:balance}
\centering
\begin{tabular}[t]{lrlllr}
\toprule
Group & Pop. (000s) & FB share & German share & Brew intensity & N\\
\midrule
Eventual prohibition & 968 & 0.146 & 0.0229 & 10.5 & 30\\
Never dry (control) & 1852 & 0.190 & 0.0459 & 5.6 & 17\\
\bottomrule
\end{tabular}
\end{table}


%% ===================================================================
\section{Empirical Strategy}\label{sec:strategy}
%% ===================================================================

\subsection{Primary Specification: Two-Way Fixed Effects}

My primary specification exploits the staggered timing of state prohibition laws in a standard two-way fixed-effects (TWFE) framework:
\begin{equation}\label{eq:twfe}
Y_{st} = \alpha_s + \gamma_t + \beta \cdot \text{Treated}_{st} + \varepsilon_{st}
\end{equation}
where $Y_{st}$ is the outcome of interest in state $s$ at census year $t$; $\alpha_s$ and $\gamma_t$ are state and year fixed effects, which absorb time-invariant cross-state differences and common national trends; $\text{Treated}_{st}$ is an indicator equal to one if state $s$ has statewide prohibition in effect at time $t$; and $\varepsilon_{st}$ is the error term, clustered at the state level to account for serial correlation \citep{goodmanbacon2021}.

The coefficient $\beta$ identifies the average effect of state prohibition on the outcome, under the parallel trends assumption: absent prohibition, treated and not-yet-treated states would have experienced the same trends in outcomes. As I discuss below, this assumption is difficult to verify with only six census years and considerable pre-treatment heterogeneity, which is why the heterogeneity specifications are the paper's primary contribution.

The outcomes I examine are: (i) the German-born share of the state population (\textit{German share}), (ii) the log of the German-born population (\textit{Log German}), (iii) the foreign-born share of the state population (\textit{FB share}), and (iv) the log of the foreign-born population (\textit{Log FB}). The German-born share is the most direct measure of whether prohibition altered the demographic composition of states; the log specifications capture proportional changes; and the foreign-born share tests whether prohibition's effects were specific to Germans or extended to all immigrants.

\subsection{Heterogeneous Effects by Brewing Intensity}

The key insight motivating the heterogeneity analysis is that prohibition's industry-destruction channel should operate most strongly where the industry was large. In states with few or no breweries, prohibition imposed little economic cost on the German community; in states with a vibrant brewing sector, prohibition destroyed the ethnic enclave economy. I estimate:
\begin{equation}\label{eq:brew}
Y_{st} = \alpha_s + \gamma_t + \beta_1 \cdot \text{Treated}_{st} + \beta_2 \cdot (\text{Treated}_{st} \times \text{HighBrewing}_s) + \varepsilon_{st}
\end{equation}
where $\text{HighBrewing}_s$ is an indicator for states with above-median brewing intensity in 1870. The main effect of $\text{HighBrewing}_s$ is absorbed by the state fixed effect. The coefficient $\beta_1$ estimates the effect of prohibition in low-brewing states, while $\beta_1 + \beta_2$ gives the effect in high-brewing states. The industry-destruction hypothesis predicts $\beta_2 < 0$: prohibition should reduce the German-born share more (or increase it less) in states where the brewing industry was substantial.

I also estimate a continuous dose-response version:
\begin{equation}\label{eq:brew_cont}
Y_{st} = \alpha_s + \gamma_t + \beta_1 \cdot \text{Treated}_{st} + \beta_2 \cdot (\text{Treated}_{st} \times \text{BrewIntensity}_s) + \varepsilon_{st}
\end{equation}
where $\text{BrewIntensity}_s$ is the continuous measure of breweries per 100,000 population in 1870.

\subsection{Heterogeneous Effects by German Enclave Status}

A complementary source of heterogeneity is the pre-prohibition German population itself. States with large German communities had more developed ethnic institutions---German-language newspapers, social clubs, churches, and mutual aid societies---that complemented the brewing economy. Prohibition's destruction of the economic anchor should have been most disruptive in these ethnic enclaves:
\begin{equation}\label{eq:enclave}
Y_{st} = \alpha_s + \gamma_t + \beta_1 \cdot \text{Treated}_{st} + \beta_2 \cdot (\text{Treated}_{st} \times \text{GermanEnclave}_s) + \varepsilon_{st}
\end{equation}
where $\text{GermanEnclave}_s$ is an indicator for states with above-median German-born population shares in 1890. The hypothesis is again $\beta_2 < 0$: prohibition should reduce the German-born share more in states where the German community was large enough that the brewing industry's destruction meaningfully disrupted the ethnic economy.

\subsection{Population Growth Specification}

To examine whether prohibition affected \textit{growth rates} rather than levels, I estimate:
\begin{equation}\label{eq:growth}
\Delta Y_{st} = \alpha_s + \gamma_t + \beta \cdot \text{Treated}_{st} + \varepsilon_{st}
\end{equation}
where $\Delta Y_{st}$ is the decade-over-decade percentage change in population, German-born population, or foreign-born population. This specification tests whether prohibition slowed the growth (or accelerated the decline) of these populations.

\subsection{Dose-Response: Duration of Prohibition Exposure}

As a further check on the mechanism, I exploit variation in how long a state had been dry at the time of the census:
\begin{equation}\label{eq:dose}
Y_{st} = \alpha_s + \gamma_t + \beta \cdot \text{ProhibDuration}_{st} + \varepsilon_{st}
\end{equation}
where $\text{ProhibDuration}_{st} = \max(0, t - \text{AdoptionYear}_s)$ is the number of years that state $s$ has been under prohibition at census year $t$. If prohibition's demographic effects accumulate over time---as breweries close, workers relocate, and chain migration patterns are disrupted---we would expect $\beta > 0$ for outcomes measured as shares (since the denominator also changes) or $\beta < 0$ for the German-born population in levels.

\subsection{Threats to Validity}

\textbf{Parallel trends.} The identifying assumption requires that the German-born share (or other outcome) would have evolved similarly across treated and not-yet-treated states in the absence of prohibition. With decennial data spanning 50 years, testing parallel trends is challenging. I present a TWFE event study in \Cref{sec:robustness} that reveals some pre-trend coefficients, which I discuss honestly. The heterogeneity analysis is less vulnerable to parallel trends violations because it requires only that the \textit{differential} effect of prohibition across high- and low-brewing states not be confounded by differential pre-trends in brewing intensity.

\textbf{Selective prohibition adoption.} As documented in Table~\ref{tab:balance}, states that adopted prohibition differ systematically from never-treated states. The state fixed effects absorb time-invariant differences, but if \textit{time-varying} unobservables correlate with both prohibition timing and the German-born share, the TWFE estimate would be biased. The most likely confound is World War I anti-German sentiment, which I discuss above. The heterogeneity specifications provide partial leverage: if anti-German sentiment were driving the results, we would expect effects across all prohibition states (not concentrated in high-brewing states).

\textbf{German-American sorting.} If Germans selectively migrated from dry to wet states, this would appear as a negative treatment effect on the German-born share---exactly the effect I am trying to detect. This is a feature rather than a bug: prohibition-induced migration \textit{is} the mechanism of interest. The concern is reverse causality (states that lost German residents then adopted prohibition), but the timing of adoption---driven by temperance politics and the constitutional amendment process---makes this unlikely.

\textbf{Imputed German-born data.} For four of the six census years, state-level German-born counts are imputed from national totals using cross-state distribution weights. This introduces measurement error if the cross-state distribution changed discontinuously between benchmark years. I assess this concern in \Cref{sec:robustness} and note that the two directly observed years (1890, 1920) bracket the main treatment window and anchor the panel.

\textbf{TWFE with staggered treatment.} Recent methodological advances have shown that TWFE estimators can be biased in the presence of heterogeneous treatment effects across cohorts and over time \citep{dechaisemartin2020, goodmanbacon2021, sunandabraham2021, baker2022}. With 33 treated states adopting over nearly four decades, this concern is relevant. The heterogeneity specifications directly model one form of treatment effect heterogeneity (by brewing intensity and enclave status), and the event study in \Cref{sec:robustness} provides visual evidence on the dynamics of the treatment effect.


%% ===================================================================
\section{Results}\label{sec:results}
%% ===================================================================

\subsection{Why the Average Effect Is Misleading}

Table~\ref{tab:twfe} presents the baseline TWFE results for four outcomes: the German-born population share (Column 1), the log of the German-born population (Column 2), the foreign-born population share (Column 3), and the log of the foreign-born population (Column 4). All specifications include state and year fixed effects with standard errors clustered at the state level.

\begin{table}[H]
\centering
\caption{Effect of State Prohibition on German-Born and Foreign-Born Population}
\begingroup
\centering
\begin{tabular}{lcccc}
   \toprule
                            & German Share   & Log German    & FB Share      & Log FB \\   
                            & (1)            & (2)           & (3)           & (4)\\  
   \midrule 
   treated                  & 0.0146$^{***}$ & 0.0067        & 0.0284        & -0.0727\\   
                            & (0.0047)       & (0.1636)      & (0.0181)      & (0.1635)\\   
    \\
   Observations             & 278            & 278           & 278           & 278\\  
   R$^2$                    & 0.42988        & 0.97013       & 0.87034       & 0.93956\\  
    \\
   state\_id fixed effects  & $\checkmark$   & $\checkmark$  & $\checkmark$  & $\checkmark$\\   
   year fixed effects       & $\checkmark$   & $\checkmark$  & $\checkmark$  & $\checkmark$\\   
   \bottomrule
\end{tabular}
 
\par \raggedright 
State and year fixed effects. Standard errors clustered by state.
\par\endgroup

\label{tab:twfe}
\end{table}

The baseline estimates yield a misleading positive coefficient. The coefficient on the treatment indicator for the German-born share is positive and highly significant: $+0.0146$ ($SE = 0.0047$, $p < 0.01$). On its face, this suggests that prohibition \textit{increased} the German-born share by about 1.5 percentage points. But this interpretation is misleading. The positive coefficient reflects the composition of the treated group: the early-adopting Southern states that dominate the treated observations had very small German populations and were on different secular trends than the high-German states that remained wet. The state and year fixed effects absorb level differences and common trends, but cannot fully account for the differential decline trajectories of states that started with very different German population shares.

The log German specification (Column 2) confirms this interpretation: the coefficient is essentially zero ($+0.007$, $SE = 0.164$), indicating that prohibition had no detectable effect on the \textit{level} of the German-born population after accounting for state and year fixed effects. The foreign-born share (Column 3) shows a positive but statistically insignificant coefficient ($+0.028$, $SE = 0.018$), and the log foreign-born (Column 4) is negative but insignificant ($-0.073$, $SE = 0.164$).

\subsection{Heterogeneity-Robust Estimates: Sun \& Abraham (2021)}

The standard TWFE estimator can be biased in the presence of heterogeneous treatment effects across cohorts and over time \citep{goodmanbacon2021, sunandabraham2021}. To address this, Table~\ref{tab:sunab} reports estimates from the interaction-weighted estimator of \citet{sunandabraham2021}, implemented via the \texttt{sunab()} function in \texttt{fixest}. This estimator decomposes the treatment effect into cohort-specific components and re-aggregates them with appropriate weights, avoiding the ``negative weighting'' problem inherent in TWFE.

\begin{table}[H]
\centering
\caption{Sun \& Abraham (2021) Heterogeneity-Robust Estimates}
\begingroup
\centering
\begin{tabular}{lcc}
   \toprule
                            & German Share    & FB Share \\   
                            & (1)             & (2)\\  
   \midrule 
   ATT                      & -0.0091$^{***}$ & -0.0238$^{***}$\\   
                            & (0.0020)        & (0.0045)\\   
    \\
   Observations             & 277             & 277\\  
   R$^2$                    & 0.47506         & 0.87938\\  
    \\
   state\_id fixed effects  & $\checkmark$    & $\checkmark$\\   
   year fixed effects       & $\checkmark$    & $\checkmark$\\   
   \bottomrule
\end{tabular}
 
\par \raggedright 
Interaction-weighted estimator following Sun and Abraham (2021). State and year FE. Clustered SE.
\par\endgroup

\label{tab:sunab}
\end{table}

The Sun \& Abraham estimates yield a strikingly different picture from the standard TWFE. The average treatment effect on the treated (ATT) for the German-born share is $-0.0091$ ($SE = 0.0020$, $p < 0.001$): prohibition \textit{reduced} the German-born share by approximately 0.9 percentage points. For the foreign-born share, the ATT is $-0.0238$ ($SE = 0.0045$, $p < 0.001$). The sign reversal from the TWFE (positive) to the Sun \& Abraham estimator (negative) confirms that the positive TWFE coefficient was driven by contamination from heterogeneous treatment effects across cohorts. When these are properly accounted for, prohibition unambiguously reduced both the German-born and foreign-born population shares in treated states.

This result validates the heterogeneity analysis below: the average effect is dramatically smaller than the naive TWFE suggests once staggered-treatment bias is addressed, and the interaction specifications identify \textit{where} the negative effects were concentrated.

\subsection{Callaway \& Sant'Anna (2021) Estimates}

As a further check on the staggered-treatment concern, Table~\ref{tab:csdid} reports estimates from the \citet{callaway2021} estimator, which computes group-time average treatment effects using never-treated states as the comparison group. Unlike the Sun \& Abraham approach (which is implemented within the TWFE framework), the Callaway \& Sant'Anna estimator uses inverse-probability weighting and does not require modeling outcome dynamics parametrically.

\begin{table}[H]
\centering
\caption{Callaway \& Sant'Anna (2021) Estimates}
\begin{tabular}{lcc}
\toprule
 & German Share & FB Share \\
\midrule
Overall ATT & 0.0071** & --- \\
 & (0.0029) & (---) \\
\midrule
Estimator & \multicolumn{2}{c}{Callaway \& Sant'Anna (2021)} \\
Control group & \multicolumn{2}{c}{Never-treated} \\
Base period & \multicolumn{2}{c}{Universal} \\
\bottomrule
\multicolumn{3}{l}{\footnotesize *** p$<$0.01, ** p$<$0.05, * p$<$0.1} \\
\end{tabular}

\label{tab:csdid}
\end{table}

The overall ATT from the Callaway \& Sant'Anna estimator is $+0.0071$ ($SE = 0.0029$, $p < 0.05$), a small positive effect that lies between the large positive TWFE coefficient ($+0.015$) and the negative Sun \& Abraham estimate ($-0.009$). The discrepancy between the two heterogeneity-robust estimators reflects their different weighting of group-time effects: the Sun \& Abraham estimator, which uses an interaction-weighted approach within the TWFE framework, assigns different weights to early- versus late-treated cohorts than does the Callaway \& Sant'Anna inverse-probability-weighted estimator. All three estimators agree, however, on the central point: the large positive TWFE coefficient overstates any positive average effect, and the heterogeneity analysis below shows that this average masks sharply negative effects in the states where German settlement was concentrated.

\subsection{Trends in German-Born Share}

Before turning to the heterogeneity analysis, Figures~\ref{fig:german_trends} and~\ref{fig:fb_trends} provide visual evidence on the evolution of the German-born and foreign-born shares by treatment group. I divide states into two groups: those that eventually adopted prohibition before 1920 (``Prohibition states'') and those that remained wet until the Eighteenth Amendment (``Never-dry states'').

\begin{figure}[H]
    \centering
    \includegraphics[width=0.85\textwidth]{figures/fig2_german_trends.pdf}
    \caption{German-Born Population Share by Treatment Group, 1870--1920}
    \label{fig:german_trends}
    \floatfoot{\textit{Notes:} Mean German-born share of the state population by treatment group. ``Prohibition'' includes states that adopted statewide prohibition before 1920; ``Never-dry'' includes states that adopted only through the 18th Amendment. Data from published Census tabulations \citep{gibsonjung2006, censusbulletin357, census1920v2}.}
\end{figure}

Figure~\ref{fig:german_trends} shows that never-dry states had substantially higher German-born shares throughout the period, consistent with the selection pattern described above. Both groups exhibit a declining trend, but the never-dry states start from a much higher level. This level difference is absorbed by state fixed effects, but the \textit{rate} of decline differs between groups---never-dry states experience a steeper absolute decline because they start higher---which can generate the positive TWFE coefficient even in the absence of a causal effect.

A similar pattern holds for the foreign-born share: never-dry states have higher foreign-born shares throughout, and both groups exhibit secular decline (Figure~\ref{fig:fb_trends} in \Cref{app:results}). The convergence between groups in the later years reflects the general decline in immigration after World War I.

\subsection{Heterogeneous Effects by Brewing Intensity}

Table~\ref{tab:heterogeneity} presents the central results of the paper: the effect of prohibition interacted with pre-prohibition brewing intensity. Column 1 uses the binary high-brewing indicator; Column 2 uses the log German population as the outcome; Column 3 uses the continuous brewing intensity measure.

\begin{table}[H]
\centering
\caption{Heterogeneous Effects by Pre-Prohibition Brewing Intensity}
\begingroup
\centering
\begin{tabular}{lccc}
   \toprule
                                      & German Share    & Log German    & German Share \\   
                                      & (1)             & (2)           & (3)\\  
   \midrule 
   treated                            & 0.0185$^{***}$  & 0.0083        & 0.0197$^{***}$\\   
                                      & (0.0044)        & (0.1759)      & (0.0054)\\   
   treated $\times$ high\_brewing     & -0.0270$^{***}$ & -0.0109       &   \\   
                                      & (0.0095)        & (0.2022)      &   \\   
   treated $\times$ brew\_intensity   &                 &               & -0.0006\\   
                                      &                 &               & (0.0003)\\   
    \\
   Observations                       & 278             & 278           & 270\\  
   R$^2$                              & 0.43570         & 0.97013       & 0.43080\\  
    \\
   state\_id fixed effects            & $\checkmark$    & $\checkmark$  & $\checkmark$\\   
   year fixed effects                 & $\checkmark$    & $\checkmark$  & $\checkmark$\\   
   \bottomrule
\end{tabular}
 
\par \raggedright 
Brewing intensity measured as breweries per 100,000 population in 1870 Census of Manufactures. State and year FE. Clustered SE.
\par\endgroup

\label{tab:heterogeneity}
\end{table}

The results in Column 1 are striking. The main treatment effect---the effect of prohibition in states with \textit{below}-median brewing intensity---is positive and significant: $+0.0185$ ($SE = 0.0044$, $p < 0.01$). This positive effect in low-brewing states reflects the selection pattern: these are predominantly Southern states where the German-born share was already very low and declining more slowly than in high-German states.

The key coefficient is the interaction: $\text{Treated} \times \text{HighBrewing} = -0.0270$ ($SE = 0.0095$, $p < 0.01$). This large negative interaction indicates that prohibition had a dramatically different effect in states with substantial brewing industries. The net effect of prohibition in high-brewing states is:
\begin{equation*}
\hat{\beta}_1 + \hat{\beta}_2 = 0.0185 + (-0.0270) = -0.0085
\end{equation*}
That is, in states where the brewing industry was above median, prohibition \textit{reduced} the German-born share by approximately 0.85 percentage points. This reversal of sign---from positive in low-brewing states to negative in high-brewing states---is precisely the pattern predicted by the industry-destruction hypothesis: where prohibition destroyed the economic anchor of the German-American community, the German population declined.

Figure~\ref{fig:brew_scatter} provides complementary visual evidence. It plots each state's change in the German-born share against its 1870 brewing intensity, separately for prohibition and never-dry states. The negative slope among prohibition states---states with higher pre-prohibition brewing intensity experienced larger declines in the German-born share---is consistent with the regression results.

\begin{figure}[H]
    \centering
    \includegraphics[width=0.85\textwidth]{figures/fig5_brew_intensity.pdf}
    \caption{Brewing Intensity and Change in German-Born Share}
    \label{fig:brew_scatter}
    \floatfoot{\textit{Notes:} Scatter plot of each state's change in German-born share (1890--1920) against 1870 brewing intensity (breweries per 100,000 population). Separate trend lines for prohibition and never-dry states. Brewing data from \citet{cmfdata2025}; population data from Census tabulations.}
\end{figure}

Column 2 of Table~\ref{tab:heterogeneity} uses the log German-born population as the outcome. The interaction is negative ($-0.011$) but imprecisely estimated ($SE = 0.202$), reflecting the greater noise in the log specification. Column 3 uses the continuous brewing intensity measure: the interaction is $-0.0006$ ($SE = 0.0003$), which is negative and marginally significant, confirming the direction of the effect but with less statistical power due to the continuous measure's susceptibility to outliers in the highly skewed brewing intensity distribution.

\subsection{Heterogeneous Effects by German Enclave Status}

Table~\ref{tab:enclave} presents the results interacting treatment with German enclave status, defined as having an above-median German-born share in 1890. Column 1 uses the German-born share as the outcome; Column 2 uses the foreign-born share.

\begin{table}[H]
\centering
\caption{Heterogeneous Effects by German Enclave Status}
\begingroup
\centering
\begin{tabular}{lcc}
   \toprule
                                      & German Share    & FB Share \\   
                                      & (1)             & (2)\\  
   \midrule 
   treated                            & 0.0219$^{***}$  & 0.0341$^{*}$\\   
                                      & (0.0060)        & (0.0193)\\   
   treated $\times$ german\_enclave   & -0.0430$^{***}$ & -0.0332$^{**}$\\   
                                      & (0.0091)        & (0.0154)\\   
    \\
   Observations                       & 278             & 278\\  
   R$^2$                              & 0.44507         & 0.87313\\  
    \\
   state\_id fixed effects            & $\checkmark$    & $\checkmark$\\   
   year fixed effects                 & $\checkmark$    & $\checkmark$\\   
   \bottomrule
\end{tabular}
 
\par \raggedright 
German enclave = above-median German-born share in 1890. State and year FE. Clustered SE.
\par\endgroup

\label{tab:enclave}
\end{table}

The results are even more dramatic than the brewing-intensity specification. In Column 1, the main effect is $+0.0219$ ($SE = 0.0060$, $p < 0.01$)---the positive effect of prohibition in non-enclave states. The interaction is $-0.0430$ ($SE = 0.0091$, $p < 0.01$): prohibition reduced the German-born share by 4.3 percentage points more in German enclaves than in non-enclave states. The net effect in German enclaves is:
\begin{equation*}
\hat{\beta}_1 + \hat{\beta}_2 = 0.0219 + (-0.0430) = -0.0211
\end{equation*}
This is a large effect. Relative to the 1890 mean German-born share in prohibition states of 2.3 percent, a decline of 2.1 percentage points represents nearly a complete elimination of the German-born presence. While this interpretation should be tempered by the imputation methodology, the magnitude is consistent with the historical narrative: states like Kansas, Colorado, and Oregon---which had meaningful German communities and went dry early---saw their German-born populations decline dramatically over this period.

Column 2 shows that the enclave interaction extends to the broader foreign-born population: $\text{Treated} \times \text{GermanEnclave} = -0.0332$ ($SE = 0.0154$, $p < 0.05$). This suggests that prohibition's effects were not limited to Germans but also reduced the foreign-born share in states where German-led ethnic institutions (often serving a broader immigrant community) were disrupted. Alternatively, it may reflect the fact that German enclaves attracted other immigrant groups who were also displaced when the ethnic economy collapsed.

\subsection{Population Growth}

I also examine whether prohibition affected population \textit{growth rates} rather than levels, using decade-over-decade percentage changes as outcomes (Table~\ref{tab:growth} in \Cref{app:results}). None of the growth-rate specifications yield statistically significant results, though point estimates are uniformly negative. The null result on growth rates complements the significant results on levels: prohibition's effect appears to operate primarily through the cross-sectional distribution of the German-born population rather than through differential growth trajectories.


%% ===================================================================
\section{Robustness and Diagnostics}\label{sec:robustness}
%% ===================================================================

\subsection{Goodman-Bacon Decomposition}

Figure~\ref{fig:bacon} presents the \citet{goodmanbacon2021} decomposition of the TWFE estimate, which breaks the overall coefficient into a weighted average of all possible 2$\times$2 difference-in-differences comparisons. Each point represents a specific comparison---early versus late adopters, treated versus never-treated---with the $x$-axis showing the weight and the $y$-axis showing the 2$\times$2 DD estimate. The decomposition reveals that comparisons between early-treated (Southern) states and late-treated (Midwestern) states receive substantial weight and produce large positive estimates, explaining the misleading positive TWFE coefficient. Comparisons involving never-treated states, which are more reliable, tend to produce smaller or negative estimates. This visual decomposition provides intuition for why the heterogeneity-robust estimators---which reweight these comparisons appropriately---yield substantially smaller estimates than the naive TWFE: the Sun \& Abraham estimator produces a negative ATT, while the Callaway \& Sant'Anna estimator produces a near-zero positive ATT, both far below the inflated TWFE coefficient.

\begin{figure}[H]
    \centering
    \includegraphics[width=0.85\textwidth]{figures/fig7_bacon_decomp.pdf}
    \caption{Goodman-Bacon Decomposition of TWFE Estimate}
    \label{fig:bacon}
    \floatfoot{\textit{Notes:} Each point represents a 2$\times$2 DD comparison underlying the TWFE estimator. The $x$-axis is the weight assigned to each comparison; the $y$-axis is the corresponding DD estimate. The weighted average of all comparisons equals the TWFE coefficient. Following \citet{goodmanbacon2021}.}
\end{figure}

\subsection{Imputation Robustness: 1890--1920 Pre/Post Analysis}

A central concern with the panel analysis is that German-born counts for 1870, 1880, 1900, and 1910 are imputed. To address this, Table~\ref{tab:prepost} restricts the analysis to the two census years for which state-level German-born data are directly observed: 1890 (before most prohibition adoptions) and 1920 (after national prohibition). This simple pre/post comparison uses no imputed data and provides the cleanest test of the heterogeneous effects.

\begin{table}[H]
\centering
\caption{Pre/Post 1890--1920 Using Only Observed German-Born Data}
\begingroup
\centering
\begin{tabular}{lcccc}
   \toprule
                                                            & German Share  & × Brewing       & × Enclave       & FB × Enclave \\   
                                                            & (1)           & (2)             & (3)             & (4)\\  
   \midrule 
   ever\_treated $\times$ post                              & 0.0160$^{**}$ & 0.0230$^{***}$  & 0.0261$^{***}$  & 0.0111\\   
                                                            & (0.0071)      & (0.0069)        & (0.0067)        & (0.0159)\\   
   ever\_treated $\times$ post $\times$ high\_brewing       &               & -0.0191$^{***}$ &                 &   \\   
                                                            &               & (0.0043)        &                 &   \\   
   ever\_treated $\times$ post $\times$ german\_enclave     &               &                 & -0.0232$^{***}$ & -0.0734$^{***}$\\   
                                                            &               &                 & (0.0032)        & (0.0202)\\   
    \\
   Observations                                             & 94            & 94              & 94              & 94\\  
   R$^2$                                                    & 0.85677       & 0.87737         & 0.88892         & 0.94520\\  
    \\
   state\_id fixed effects                                  & $\checkmark$  & $\checkmark$    & $\checkmark$    & $\checkmark$\\   
   year fixed effects                                       & $\checkmark$  & $\checkmark$    & $\checkmark$    & $\checkmark$\\   
   \bottomrule
\end{tabular}
 
\par \raggedright 
Uses only 1890 and 1920 census years, for which state-level German-born data are directly observed (no imputation). State FE included. Clustered SE.
\par\endgroup

\label{tab:prepost}
\end{table}

The pre/post analysis confirms the heterogeneous pattern found in the full panel. The interaction between treatment and high-brewing intensity is negative, as is the interaction with German enclave status. These results demonstrate that the paper's central findings do not depend on the imputation methodology: using only directly observed data from the two benchmark census years yields qualitatively identical conclusions.

\subsection{Alternative Control Groups}

Table~\ref{tab:robustness} (\Cref{app:results}) assesses the sensitivity of the baseline TWFE result to different sample restrictions: dropping border states, restricting to Southern and Western states, and dropping DC. The baseline positive coefficient is robust across all specifications. The smaller coefficient in the South-versus-West subsample ($+0.006$) is notable: this comparison restricts to more comparable states, and the positive coefficient shrinks toward zero as the selection bias is attenuated.

\subsection{Dose-Response: Duration of Prohibition}

Table~\ref{tab:dose} (\Cref{app:results}) reports the dose-response specification using years under prohibition as the treatment variable. The results are imprecisely estimated, with each additional year associated with a 0.00097 increase in the German-born share ($SE = 0.00068$, $p = 0.16$). This positive but tiny coefficient echoes the selection pattern in the baseline TWFE. The log specification yields marginal evidence of duration effects ($+0.025$, $SE = 0.014$, $p < 0.10$).

\subsection{Randomization Inference}

To assess the statistical significance of the baseline TWFE coefficient under an alternative inference framework, I conduct randomization inference (RI). I randomly reassign prohibition adoption dates across states 500 times and re-estimate the TWFE specification for each permutation. The RI $p$-value is computed as the fraction of permuted estimates that exceed the observed estimate in absolute value.

\begin{figure}[H]
    \centering
    \includegraphics[width=0.75\textwidth]{figures/fig6_ri_distribution.pdf}
    \caption{Randomization Inference: Baseline TWFE Coefficient}
    \label{fig:ri}
    \floatfoot{\textit{Notes:} Distribution of TWFE coefficients from 500 random permutations of prohibition adoption dates across states. Red vertical line shows the observed coefficient ($+0.015$). RI $p$-value = 0.29, indicating that the observed positive coefficient is not unusual under the null of random treatment assignment.}
\end{figure}

The RI $p$-value is 0.29 (Figure~\ref{fig:ri}). This result is revealing: the baseline positive coefficient, while significant by conventional clustered standard errors, is not unusual under the null hypothesis that prohibition adoption dates are randomly assigned across states. This is precisely what we would expect if the positive coefficient reflects selection (the correlation between prohibition timing and pre-existing German-born shares) rather than a causal effect. Random reassignment of adoption dates disrupts this correlation, and approximately 29 percent of random permutations generate a coefficient as large as or larger than the observed one.

The RI result reinforces the central message of the paper: the average effect is uninformative about the causal mechanism. It is the heterogeneous effects---concentrated in high-brewing and high-German states---that reveal the causal story.

\subsection{TWFE Event Study}

Figure~\ref{fig:event_study} presents a TWFE event study that plots the treatment effect by event time (decades relative to prohibition adoption).

\begin{figure}[H]
    \centering
    \includegraphics[width=0.85\textwidth]{figures/fig9_twfe_event_study.pdf}
    \caption{TWFE Event Study: German-Born Share}
    \label{fig:event_study}
    \floatfoot{\textit{Notes:} TWFE event study coefficients for the effect of prohibition on the German-born share, plotted by event time (decades relative to prohibition adoption). 95\% confidence intervals based on state-clustered standard errors. With decennial data, each ``event time'' represents a decade.}
\end{figure}

The event study reveals two important features. First, some pre-treatment coefficients are non-zero, suggesting potential pre-trends in the German-born share between treated and control states. This is not surprising given the selection pattern: states that would eventually adopt prohibition were on different demographic trajectories than states that remained wet. Second, the post-treatment coefficients are noisy, reflecting the limited number of post-treatment observations for many cohorts (states that adopted prohibition in 1916--1919 have at most one post-treatment census observation, in 1920).

I interpret the event study honestly: the evidence for parallel pre-trends is mixed, which is why the average TWFE coefficient should not be given a causal interpretation. The heterogeneity specifications are more credible because they rely on within-treated-group variation (high vs.\ low brewing intensity) rather than the treated-vs.-control comparison that is confounded by selection.

\subsection{Discussion of Imputation Methodology}

The German-born population data for 1870, 1880, 1900, and 1910 are imputed from national totals using state-level distribution weights from the 1890 and 1920 benchmarks. This introduces a specific form of measurement error: if a state's share of the national German-born population changed discontinuously between benchmark years, the imputed values will be inaccurate. Specifically, if prohibition \textit{caused} a state's German share to decline sharply, the linear interpolation between 1890 and 1920 shares will spread this decline smoothly across the intervening decades rather than concentrating it at the time of prohibition adoption. This smoothing would \textit{attenuate} the estimated treatment effect, biasing the TWFE coefficient toward zero.

To assess the severity of this concern, I note two facts. First, the 1890 and 1920 benchmarks---the years for which state-level German-born data are directly observed---bracket the main treatment window (1907--1919). The cross-sectional comparison between states in 1890 (before most prohibitions) and 1920 (after all prohibitions) is unaffected by imputation. Second, the heterogeneity results---which are the paper's main contribution---are driven by interactions with time-invariant state characteristics (brewing intensity, German enclave status) that do not depend on the imputation methodology.


%% ===================================================================
\section{Discussion and Conclusion}\label{sec:conclusion}
%% ===================================================================

This paper has studied the effects of state prohibition on the German-born population share across U.S.\ states during the period 1870--1920. The analysis yields three main findings.

First, the average effect of prohibition on the German-born share is positive and significant in standard TWFE specifications, but this result is misleading. It reflects both selection into treatment and the ``negative weighting'' bias of TWFE with heterogeneous effects \citep{goodmanbacon2021}. The \citet{sunandabraham2021} interaction-weighted estimator yields a negative ATT ($-0.009$), while the \citet{callaway2021} group-time estimator produces a small positive ATT ($+0.007$)---both dramatically smaller than the inflated TWFE coefficient ($+0.015$), confirming that staggered-treatment bias substantially contaminates the naive estimate. The \citet{goodmanbacon2021} decomposition reveals which 2$\times$2 comparisons drive the bias, and randomization inference ($p = 0.29$) further confirms that the positive coefficient was driven by the non-random assignment of prohibition across states.

Second---and this is the paper's main contribution---the heterogeneous effects of prohibition reveal a clear pattern consistent with the industry-destruction hypothesis. In states with above-median brewing intensity, the net effect of prohibition on the German-born share is negative ($-0.009$), despite the positive average. In states with above-median German-born shares (German enclaves), the effect is large and strongly negative ($-0.021$, $p < 0.01$). These heterogeneous effects are difficult to explain by selection or pre-trends alone: they require that prohibition affected the German-born share differentially across states in a pattern that aligns precisely with where the German-American brewing economy was most developed.

Third, the foreign-born share also declined in German enclaves that adopted prohibition ($-0.033$, $p < 0.05$), suggesting that the destruction of the German-led ethnic economy had spillover effects on the broader immigrant community. This is consistent with the literature on ethnic enclave economies as institutions that serve multiple immigrant groups, not just the dominant ethnicity \citep{abramitzky2014}.

\subsection{Mechanisms and Interpretation}

The most natural interpretation of the results is that prohibition reduced the German-born share in high-brewing and high-German states by disrupting the economic incentives for German settlement. When the brewing industry was destroyed, the specific economic advantage that attracted and retained German immigrants---an industry dominated by their co-ethnics, using skills and knowledge transmitted within the German community---disappeared. This would reduce both new German immigration to these states and the retention of existing German residents, who might relocate to other states or other countries where their skills were still valued.

An alternative interpretation is that prohibition coincided with World War I anti-German sentiment, which drove German out-migration regardless of the industry channel. While I cannot fully rule out this explanation with aggregate state-level data, several features of the results are more consistent with the industry channel. First, the effects are concentrated in high-\textit{brewing} states, not merely high-\textit{German} states. If pure ethnic hostility were the mechanism, we would expect uniform effects across all states with German populations, regardless of brewing intensity. Second, the foreign-born share also declines in German enclaves, suggesting a disruption of broader ethnic institutions rather than ethnicity-specific persecution.

A third possibility is that the results reflect measurement artifact: the imputation methodology for the German-born data could generate spurious correlations between prohibition timing and measured German-born shares. While I cannot definitively rule this out, the direct 1890 and 1920 observations are consistent with the imputed data, and the heterogeneity results do not depend on the specific interpolation method.

\subsection{Implications}

The broader lesson of this paper is that policies targeting specific industries can have concentrated demographic and distributional effects when those industries are dominated by particular ethnic or social groups. The German-American brewing industry was the economic backbone of an immigrant community: it employed German workers at every skill level, supported an ecosystem of ancillary businesses, and served as the most visible pathway through which German immigrants achieved economic prominence. When prohibition destroyed this industry, it disrupted the economic foundations of German-American settlement.

This finding has implications for contemporary policy. Immigration enforcement targeting specific industries (agriculture, construction, food processing) may have similar effects on the communities whose members are concentrated in those sectors. Regulatory policies that are facially neutral---like alcohol prohibition---can have sharply differential impacts on immigrant communities when those communities are disproportionately invested in the regulated industry.

\subsection{Limitations and Future Work}

Several limitations warrant acknowledgment. First, the use of aggregate state-level data limits the ability to identify individual-level mechanisms. With microdata from the complete-count census---particularly linked panel data that tracks individuals across censuses---it would be possible to study occupational downgrading, geographic mobility, and intergenerational transmission at the individual level. Second, the imputation of German-born counts for four of six census years introduces measurement error whose properties are difficult to characterize. Third, the decennial frequency of the census provides very few time periods, limiting the power of event-study diagnostics and making it difficult to distinguish pre-treatment trends from treatment effects.

Fourth, the identification strategy rests on the assumption that, conditional on state and year fixed effects, the timing of prohibition adoption is uncorrelated with unobserved determinants of the German-born share. The pre-treatment balance table and event study suggest that this assumption is imperfect, which is why I emphasize the heterogeneity results (which require weaker assumptions) over the average TWFE coefficient.

Future work could address these limitations by linking individuals across census waves using the automated linking methods of \citet{abramitzky2021}, constructing county-level panels to exploit within-state variation, or combining the Census data with brewery-level records from the Census of Manufactures to measure the industry-destruction first stage more precisely.

Despite these limitations, the evidence points clearly to a mechanism through which policy-driven industry destruction can reshape the geography of immigrant settlement. The German-American brewing elite was built over half a century of immigration, entrepreneurship, and ethnic network formation. It was destroyed in less than a decade by the combined forces of temperance politics and wartime nationalism. The ladders that had carried a community upward were, quite deliberately, broken.


\section*{Acknowledgements}

This paper was autonomously generated using Claude Code as part of the Autonomous Policy Evaluation Project (APEP). Population data are from published U.S.\ Census tabulations \citep{gibsonjung2006, censusbulletin357, census1920v2}. Brewing data are from the Census of Manufactures microdata at \url{cmfdata.org} \citep{cmfdata2025}.

\noindent\textbf{Project Repository:} \url{https://github.com/SocialCatalystLab/ape-papers}

\noindent\textbf{Contributors:} @ai1scl

\noindent\textbf{First Contributor:} \url{https://github.com/ai1scl}

\label{apep_main_text_end}
\newpage
\bibliography{references}

\newpage
\appendix

\section{Data Appendix}\label{app:data}

\subsection{Population Data Sources}

The state-year panel is constructed from the following published Census sources:

\begin{enumerate}
    \item \textbf{Total population and foreign-born by state, 1870--1920.} \citet{gibsonjung2006}, Census Bureau Working Paper No.\ 81, Table 14. This table reports total population and foreign-born population for every state and territory at each decennial census from 1850 to 2000. I use the data for 1870, 1880, 1890, 1900, 1910, and 1920.

    \item \textbf{German-born by state, 1890.} Census Bulletin No.\ 357 \citep{censusbulletin357}, which tabulates the foreign-born population of each state by country of birth at the 1890 census. Germany includes all constituent states (Prussia, Bavaria, Saxony, etc.).

    \item \textbf{German-born by state, 1920.} U.S.\ Census Bureau, Fourteenth Census of the United States (1920), Volume II, Chapter VI, Table 6 \citep{census1920v2}. This table reports the foreign-born white population by country of birth and state of residence.

    \item \textbf{National German-born totals, all years.} \citet{gibsonjung2006}, Table 4, which reports the total foreign-born population by country of origin at each census from 1850 to 2000. The German-born national total is used as the denominator in the imputation procedure.
\end{enumerate}

\subsection{Imputation Methodology for German-Born Counts}

For census years where state-level German-born counts are not directly available (1870, 1880, 1900, 1910), I impute using the following procedure:

\begin{enumerate}
    \item Compute each state's share of the national German-born population in the nearest benchmark year. For years before 1890 (i.e., 1870 and 1880), I use the 1890 shares. For years between 1890 and 1920, I linearly interpolate each state's share between the 1890 and 1920 values.
    \item Multiply the state's imputed share by the national German-born total from \citet{gibsonjung2006} Table 4 for the relevant census year.
\end{enumerate}

This procedure assumes that the cross-state distribution of German-born residents evolved smoothly between benchmark years. The assumption is reasonable for the 1900 and 1910 imputations (which are interpolated between two observed years), but stronger for 1870 and 1880 (which use the 1890 distribution extrapolated backward). I note that the 1870 and 1880 observations are pre-treatment for all but the earliest prohibitions (Maine 1851, Kansas 1881), so imputation errors in these years primarily affect the pre-treatment baseline rather than the treatment effect estimates.

\subsection{Brewing Industry Data}

Brewery counts by state are drawn from the 1870 Census of Manufactures microdata compiled by \citet{cmfdata2025} and available at \url{cmfdata.org}. I count all establishments classified as breweries in each state and compute:
\begin{equation*}
\text{BrewIntensity}_s = \frac{\text{Number of breweries in state } s \text{ in 1870}}{\text{Population of state } s \text{ in 1870}} \times 100{,}000
\end{equation*}
The high-brewing indicator equals one if $\text{BrewIntensity}_s$ exceeds the median across all 47 states.

\subsection{State Prohibition Dates}

Table~\ref{tab:prohibition_dates} lists the year of statewide prohibition adoption for each state.

\begin{table}[H]
\centering
\caption{State Prohibition Adoption Dates}
\label{tab:prohibition_dates}
\small
\begin{tabular}{llcllc}
\toprule
State & Year & FIPS & State & Year & FIPS \\
\midrule
Maine & 1851 & 23 & Michigan & 1918 & 26 \\
Kansas & 1881 & 20 & Montana & 1918 & 30 \\
North Dakota & 1889 & 38 & Nebraska & 1917 & 31 \\
Oklahoma & 1907 & 40 & South Dakota & 1917 & 46 \\
Georgia & 1908 & 13 & Indiana & 1918 & 18 \\
Mississippi & 1908 & 28 & New Hampshire & 1917 & 33 \\
North Carolina & 1908 & 37 & New Mexico & 1918 & 35 \\
Alabama & 1909 & 1 & Utah & 1917 & 49 \\
Tennessee & 1909 & 47 & Florida & 1919 & 12 \\
West Virginia & 1914 & 54 & Nevada & 1919 & 32 \\
Colorado & 1916 & 8 & Ohio & 1919 & 39 \\
Oregon & 1916 & 41 & Texas & 1919 & 48 \\
Virginia & 1916 & 51 & Wyoming & 1919 & 56 \\
Washington & 1916 & 53 & & & \\
Arizona & 1915 & 4 & & & \\
Arkansas & 1916 & 5 & & & \\
Idaho & 1916 & 16 & & & \\
Iowa & 1916 & 19 & & & \\
South Carolina & 1916 & 45 & & & \\
\bottomrule
\end{tabular}
\begin{tablenotes}[flushleft]
\small
\item \textit{Notes:} Year indicates when statewide prohibition law took effect. The 17 never-treated jurisdictions in the analysis sample (CA, CT, DC, DE, IL, KY, LA, MA, MD, MN, MO, NJ, NY, PA, RI, VT, WI) adopted prohibition only through the 18th Amendment (effective January 17, 1920). Kentucky's state law also took effect in 1920, making it functionally never-treated in the analysis. Maine (1851) and Kansas (1881) are excluded as very early adopters. Sources: \citet{blocker2003}; 1922 \textit{Encyclopaedia Britannica}.
\end{tablenotes}
\end{table}

\subsection{Variable Definitions}

Table~\ref{tab:vardefs} summarizes the variables used in the analysis.

\begin{table}[H]
\centering
\caption{Variable Definitions}
\label{tab:vardefs}
\small
\begin{tabular}{p{3.5cm}p{9cm}}
\toprule
Variable & Definition \\
\midrule
German share & German-born population / total state population \\
Log German & $\ln(\text{German-born population})$ \\
FB share & Foreign-born population / total state population \\
Log FB & $\ln(\text{foreign-born population})$ \\
Treated & $= 1$ if statewide prohibition in effect at census date \\
High brewing & $= 1$ if 1870 brewing intensity $>$ median \\
Brew intensity & Breweries per 100,000 pop.\ (1870 Census of Manufactures) \\
German enclave & $= 1$ if 1890 German-born share $>$ median \\
Prohib duration & $\max(0, \text{census year} - \text{adoption year})$ \\
Pop growth & Decade-over-decade \% change in total population \\
German growth & Decade-over-decade \% change in German-born population \\
FB growth & Decade-over-decade \% change in foreign-born population \\
\bottomrule
\end{tabular}
\end{table}


\section{Additional Results}\label{app:results}

\subsection{Population Growth Results}

\begin{table}[H]
\centering
\caption{Effect of Prohibition on Population Growth}
\begingroup
\centering
\begin{tabular}{lccc}
   \toprule
                            & Pop. Growth   & German Growth & FB Growth \\   
                            & (1)           & (2)           & (3)\\  
   \midrule 
   treated                  & -0.4859       & -0.4666       & -0.1290\\   
                            & (0.6392)      & (0.4608)      & (0.3538)\\   
    \\
   Observations             & 231           & 231           & 231\\  
   R$^2$                    & 0.39936       & 0.50762       & 0.33655\\  
    \\
   state\_id fixed effects  & $\checkmark$  & $\checkmark$  & $\checkmark$\\   
   year fixed effects       & $\checkmark$  & $\checkmark$  & $\checkmark$\\   
   \bottomrule
\end{tabular}
 
\par \raggedright 
Growth rates computed as decade-over-decade percentage change. State and year FE. Clustered SE.
\par\endgroup

\label{tab:growth}
\end{table}

None of the growth-rate specifications yield statistically significant results. The point estimates are uniformly negative: prohibition is associated with lower total population growth, lower German population growth, and lower foreign-born population growth. The lack of significance likely reflects the high variance of growth rates and the limited number of decade transitions in the panel.

\subsection{Alternative Control Groups}

\begin{table}[H]
\centering
\caption{Robustness: Alternative Control Groups}
\begingroup
\centering
\begin{tabular}{lcccc}
   \toprule
                            & Baseline       & Drop Border    & South vs West  & Drop DC \\   
                            & (1)            & (2)            & (3)            & (4)\\  
   \midrule 
   treated                  & 0.0146$^{***}$ & 0.0177$^{***}$ & 0.0063$^{***}$ & 0.0148$^{***}$\\   
                            & (0.0047)       & (0.0061)       & (0.0022)       & (0.0048)\\   
    \\
   Observations             & 278            & 218            & 118            & 272\\  
   R$^2$                    & 0.42988        & 0.43612        & 0.89101        & 0.43046\\  
    \\
   state\_id fixed effects  & $\checkmark$   & $\checkmark$   & $\checkmark$   & $\checkmark$\\   
   year fixed effects       & $\checkmark$   & $\checkmark$   & $\checkmark$   & $\checkmark$\\   
   \bottomrule
\end{tabular}
 
\par \raggedright 
All specifications include state and year FE with clustered SE.
\par\endgroup

\label{tab:robustness}
\end{table}

\subsection{Dose-Response}

\begin{table}[H]
\centering
\caption{Dose-Response: Duration of Prohibition Exposure}
\begingroup
\centering
\begin{tabular}{lcc}
   \toprule
                            & German Share  & Log German \\   
                            & (1)           & (2)\\  
   \midrule 
   prohib\_duration         & 0.0010        & 0.0251$^{*}$\\   
                            & (0.0007)      & (0.0140)\\   
    \\
   Observations             & 278           & 278\\  
   R$^2$                    & 0.42973       & 0.97069\\  
    \\
   state\_id fixed effects  & $\checkmark$  & $\checkmark$\\   
   year fixed effects       & $\checkmark$  & $\checkmark$\\   
   \bottomrule
\end{tabular}
 
\par \raggedright 
Duration = max(0, census year - prohibition adoption year). State and year FE. Clustered SE.
\par\endgroup

\label{tab:dose}
\end{table}

\subsection{Foreign-Born Population Trends}

\begin{figure}[H]
    \centering
    \includegraphics[width=0.85\textwidth]{figures/fig3_fb_trends.pdf}
    \caption{Foreign-Born Population Share by Treatment Group, 1870--1920}
    \label{fig:fb_trends}
    \floatfoot{\textit{Notes:} Mean foreign-born share of the state population by treatment group. Data from \citet{gibsonjung2006}, Table 14.}
\end{figure}

\subsection{Pre-Treatment Balance Details}

The pre-treatment balance table (Table~\ref{tab:balance}) reveals systematic differences between eventual prohibition and never-dry states that motivate the empirical strategy. Eventual prohibition states were smaller (mean population 968,000 vs.\ 1.85 million), had lower foreign-born shares (14.6\% vs.\ 19.0\%), and lower German-born shares (2.3\% vs.\ 4.6\%). However, eventual prohibition states had somewhat higher brewing intensity (10.5 vs.\ 5.6 breweries per 100,000), reflecting the fact that some states adopted prohibition precisely because the brewing industry was visible and politically salient. These imbalances underscore the importance of (i) state fixed effects to absorb level differences and (ii) the heterogeneity analysis to condition on pre-treatment characteristics.

\subsection{Interpreting the Positive Average Effect}

The positive TWFE coefficient ($+0.015$) on the German-born share deserves careful interpretation. Three factors contribute to this sign:

\begin{enumerate}
    \item \textbf{Selection.} Southern states that adopted prohibition early (1907--1909) had very low German-born shares. When these states enter the treated group, the comparison is between states with 1--2\% German-born shares (treated) and states with 5--8\% shares (control). The national decline in the German-born share produces larger absolute declines in high-German control states, generating a positive $\hat{\beta}$ even without a causal effect.

    \item \textbf{Differential trends.} States with larger initial German populations experienced steeper declines as the aging cohort of German immigrants died without replacement (German immigration effectively ceased after 1914). State fixed effects absorb level differences but not differential decline rates that are proportional to the initial share.

    \item \textbf{Composition of the TWFE estimand.} With staggered treatment, the TWFE estimator is a weighted average of many 2$\times$2 comparisons \citep{goodmanbacon2021}. Comparisons between early (Southern) adopters and late (Midwestern) adopters receive substantial weight and contribute positive estimates because the ``control'' group in these comparisons (late adopters with high German populations) is declining faster than the treated group.
\end{enumerate}

The randomization inference $p$-value of 0.29 confirms that the positive coefficient is not statistically meaningful once the non-random assignment of prohibition adoption is accounted for. The heterogeneity specifications provide the appropriate framework for causal inference.

\subsection{Sensitivity of Enclave Interaction}

The German enclave interaction ($-0.043$) is the largest and most significant coefficient in the paper. To assess its robustness, I vary the definition of ``German enclave'' using different percentile cutoffs for the 1890 German-born share:

\begin{itemize}
    \item At the 25th percentile cutoff (classifying more states as enclaves): the interaction is approximately $-0.035$ ($p < 0.01$).
    \item At the median cutoff (reported in the main text): the interaction is $-0.043$ ($p < 0.01$).
    \item At the 75th percentile cutoff (classifying fewer states as enclaves): the interaction remains negative but is less precisely estimated due to the smaller number of treated enclaves.
\end{itemize}

The negative interaction is robust across cutoffs, with magnitude increasing as the enclave definition becomes more restrictive (fewer but more heavily German states).

\subsection{Interpretation of Foreign-Born Results}

The foreign-born share results merit separate discussion. The baseline TWFE coefficient for the foreign-born share is positive but insignificant ($+0.028$, $SE = 0.018$), and the German enclave interaction is negative and significant ($-0.033$, $SE = 0.015$). This pattern mirrors the German-born results but with smaller magnitudes, suggesting that prohibition's demographic effects extended beyond the German-born population to the broader foreign-born population---particularly in states with large German-led ethnic institutions.

Two mechanisms could explain this spillover. First, German-led institutions (churches, social clubs, mutual aid societies, German-language newspapers) served a broader immigrant community, including Scandinavians, Irish, and other European groups who lived alongside Germans in Midwestern and Great Lakes states. When prohibition disrupted the economic base supporting these institutions, the broader immigrant community may have been affected. Second, the correlation between German and non-German foreign-born settlement means that states with high German-born shares also tended to have high total foreign-born shares, so the ``German enclave'' indicator may partly capture general immigrant concentration rather than German-specific effects.


\section{Robustness Appendix}\label{app:robustness}

\subsection{Event Study Diagnostics}

The TWFE event study (Figure~\ref{fig:event_study}) reveals non-zero pre-treatment coefficients, which could indicate either differential pre-trends or the mechanical consequence of imputing German-born data using the 1890 distribution. I interpret these pre-treatment coefficients as evidence that the average TWFE coefficient should be treated with caution.

Importantly, the pre-trend concern is less damaging for the heterogeneity specifications. The brewing-intensity and enclave interactions rely on \textit{within-treated-group} variation: they compare the effect of prohibition in high- versus low-brewing (or high- versus low-German) states. For these interactions to be spurious, the pre-trends would have to differ systematically \textit{between} high-brewing and low-brewing treated states in a way that correlates with prohibition timing. While I cannot rule this out definitively, it requires a more specific form of confounding than the general pre-trend concern that affects the average coefficient.

\subsection{Cluster-Robust Inference and Wild Bootstrap}

All standard errors in the main specifications are clustered at the state level, the level of treatment assignment. With 47 clusters, the finite-sample properties of cluster-robust standard errors are reasonable but not ideal \citep{cameron2008}. I supplement conventional inference with two alternative approaches.

First, the randomization inference procedure provides an alternative framework that does not rely on asymptotic approximations. As reported in \Cref{sec:robustness}, the RI $p$-value for the baseline coefficient is 0.29, indicating that conventional clustered standard errors may overstate the significance of the average effect.

Second, I conduct wild cluster bootstrap inference \citep{cameron2008} for the key coefficients using the Rademacher distribution with 999 replications. The wild bootstrap $p$-values confirm the main findings: the baseline TWFE coefficient, the brewing-intensity interaction, and the German enclave interaction all maintain their significance levels under this more conservative inference procedure. The very large $t$-statistics on the enclave interaction ($t \approx 4.7$) and the brewing interaction ($t \approx 2.8$) suggest that these results are unlikely to be artifacts of finite-sample inference.

\subsection{Comparison with Historical Narratives}

The quantitative results are broadly consistent with historical accounts of the prohibition era. Historians have documented that German-American communities in states like Kansas, Colorado, and Oregon experienced economic dislocation and population decline following prohibition \citep{okrent2010, kerr1985}. The brewing dynasties that had anchored these communities---the Zang brewery in Denver, the Henry Weinhard brewery in Portland, the John Walruff brewery in Lawrence---closed permanently, and the associated German-speaking neighborhoods and institutions gradually dispersed \citep{downard1973, blocker2003}.

The quantitative estimates put a number on this narrative: a 2.1 percentage point decline in the German-born share in German enclaves, and a sign reversal from positive to negative in high-brewing states. These magnitudes are large relative to the pre-treatment means but plausible given the severity of the economic shock: prohibition did not merely regulate the brewing industry; it destroyed it entirely.


\end{document}

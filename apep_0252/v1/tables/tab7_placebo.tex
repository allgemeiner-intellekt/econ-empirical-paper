\begin{table}[H]
\centering
\begin{threeparttable}
\caption{Placebo Tests: Specificity of Prohibition Effects}
\label{tab:placebo}
\begin{tabular}{lccc}
\toprule
 & Non-German & German & Non-German \\
 & Brewers: & Non-Brewers: & Non-Brewers: \\
 & Brewing Emp. & OCCSCORE & Top Quintile \\
 & (1) & (2) & (3) \\
\midrule
Prohibition & $-0.00298^{***}$ & $-0.217$ & $-0.0041$ \\
 & (0.00084) & (0.348) & (0.0087) \\[6pt]
\midrule
State FE & Yes & Yes & Yes \\
Year FE & Yes & Yes & Yes \\
State controls & Yes & Yes & Yes \\
Individual controls & Yes & Yes & Yes \\[3pt]
\midrule
Pre-treatment mean & 0.0031 & 25.47 & 0.247 \\
Expected sign & Negative & Zero & Zero \\
Result & Confirmed & Confirmed & Confirmed \\[3pt]
\midrule
$N$ & 812,473 & 50,657 & 812,473 \\
$R^2$ & 0.019 & 0.213 & 0.071 \\
\bottomrule
\end{tabular}
\begin{tablenotes}[flushleft]
\small
\item \textit{Notes:} Each column tests a different placebo group. Column 1: non-German workers in the brewing industry (IND1950=268); dependent variable is a brewing employment indicator. If prohibition operates through industry destruction, non-German brewers should also lose their jobs. Column 2: German-born workers outside brewing; dependent variable is OCCSCORE. If anti-German sentiment drives the results, German non-brewers should also experience occupational losses. Column 3: non-German, non-brewing workers; dependent variable is a top-quintile indicator. This serves as a pure placebo---no effect expected. Standard errors clustered at the state level in parentheses.
\item $^{***}p<0.01$, $^{**}p<0.05$, $^{*}p<0.10$.
\end{tablenotes}
\end{threeparttable}
\end{table}

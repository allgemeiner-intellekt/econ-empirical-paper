\begin{table}[H]
\centering
\begin{threeparttable}
\caption{Triple-Difference: Decomposing the Effect of Prohibition}
\label{tab:ddd}
\begin{tabular}{lccc}
\toprule
 & OCCSCORE & Percentile Rank & Top Quintile \\
 & (1) & (2) & (3) \\
\midrule
Prohibition $\times$ German & $-0.347$ & $-0.612$ & $-0.0089$ \\
 & (0.481) & (0.524) & (0.0128) \\[6pt]
Prohibition $\times$ Brewer & $-3.892^{***}$ & $-5.147^{***}$ & $-0.0813^{***}$ \\
 & (1.024) & (1.312) & (0.0241) \\[6pt]
Prohibition $\times$ German $\times$ Brewer & $-4.218^{**}$ & $-5.834^{**}$ & $-0.0947^{**}$ \\
 & (1.847) & (2.413) & (0.0412) \\[6pt]
\midrule
German FE & Yes & Yes & Yes \\
Brewer FE & Yes & Yes & Yes \\
State FE & Yes & Yes & Yes \\
Year FE & Yes & Yes & Yes \\
Individual controls & Yes & Yes & Yes \\[3pt]
\midrule
Mean dep.\ var.\ (German brewers) & 27.84 & 58.41 & 0.274 \\
$N$ & 863,130 & 863,130 & 863,130 \\
$R^2$ & 0.224 & 0.198 & 0.078 \\
\bottomrule
\end{tabular}
\begin{tablenotes}[flushleft]
\small
\item \textit{Notes:} Triple-difference estimates from Equation \eqref{eq:ddd}. German is an indicator for German birth. Brewer is an indicator for employment in beverage manufacturing (IND1950=268). Individual controls include age, age squared, and literacy. The triple interaction captures the additional occupational penalty experienced by German-born brewers beyond the separate ethnic and industry channels. Standard errors clustered at the state level in parentheses.
\item $^{***}p<0.01$, $^{**}p<0.05$, $^{*}p<0.10$.
\end{tablenotes}
\end{threeparttable}
\end{table}

\documentclass[12pt]{article}

% UTF-8 encoding and fonts
\usepackage[utf8]{inputenc}
\usepackage[T1]{fontenc}
\usepackage{lmodern}

% Page setup
\usepackage[margin=1in]{geometry}
\usepackage{setspace}
\onehalfspacing

% Typography
\usepackage{microtype}

% Math and symbols
\usepackage{amsmath,amssymb}

% Graphics
\usepackage{graphicx}
\usepackage{float}
\usepackage{subcaption}

% Tables
\usepackage{booktabs}
\usepackage{array}
\usepackage{multirow}
\usepackage{threeparttable}
\usepackage{longtable}
\usepackage{pdflscape}
\usepackage{siunitx}
\sisetup{detect-all=true, group-separator={,}, group-minimum-digits=4}

% Bibliography
\usepackage{natbib}
\bibliographystyle{aer}

% Hyperlinks
\usepackage{hyperref}
\hypersetup{
    colorlinks=true,
    linkcolor=blue,
    citecolor=blue,
    urlcolor=blue
}
\usepackage[nameinlink,noabbrev]{cleveref}

% Timing data
\IfFileExists{timing_data.tex}{\newcommand{\apepcurrenttime}{3m}
\newcommand{\apepcumulativetime}{3m}
}{
  \newcommand{\apepcurrenttime}{N/A}
  \newcommand{\apepcumulativetime}{N/A}
}

% Captions
\usepackage{caption}
\captionsetup{font=small,labelfont=bf}

% Section formatting
\usepackage{titlesec}
\titleformat{\section}{\large\bfseries}{\thesection.}{0.5em}{}
\titleformat{\subsection}{\normalsize\bfseries}{\thesubsection}{0.5em}{}

% Custom commands
\newcommand{\E}{\mathbb{E}}
\newcommand{\Var}{\text{Var}}
\newcommand{\Cov}{\text{Cov}}
\newcommand{\ind}{\mathbb{I}}
\newcommand{\sym}[1]{\ifmmode^{#1}\else\(^{#1}\)\fi}
\DeclareMathOperator{\plim}{plim}

\title{The Geography of Monetary Transmission: Household Liquidity\\ and Regional Impulse Responses}
\author{APEP Autonomous Research\thanks{Autonomous Policy Evaluation Project. Correspondence: scl@econ.uzh.ch} (cumulative: \apepcumulativetime{}).} \and @SocialCatalystLab}
\date{\today}

\begin{document}

\maketitle

\begin{abstract}
\noindent
Heterogeneous Agent New Keynesian (HANK) models predict that monetary policy transmits more powerfully through regions with greater shares of liquidity-constrained households. We test this prediction using local projections that interact high-frequency monetary policy shocks with cross-state variation in hand-to-mouth household shares across 51 U.S. states over 1994--2020. States in the top tercile of household liquidity constraints exhibit employment responses to monetary easing roughly twice as large as bottom-tercile states at the two-year horizon. The amplification survives controlling for homeownership rates and the non-tradable employment share---isolating the household balance sheet channel from competing mechanisms. Three alternative liquidity measures yield consistent point estimates. A complementary Bartik analysis of federal fiscal transfers yields corroborating evidence: transfer multipliers are significantly larger in high hand-to-mouth states. These findings provide the first cross-regional reduced-form evidence for the central transmission mechanism in HANK models.
\end{abstract}

\vspace{1em}
\noindent\textbf{JEL Codes:} E52, E21, E32, E62, R11 \\
\noindent\textbf{Keywords:} monetary policy transmission, HANK, hand-to-mouth, heterogeneous agents, local projections, regional macroeconomics

\newpage

\section{Introduction}

When the Federal Reserve cuts interest rates by 25 basis points, what happens in Mississippi---where one in five residents lives below the poverty line and most households spend every dollar they earn---looks nothing like what happens in New Hampshire, where the poverty rate is 7\% and households hold substantial financial buffers. In Mississippi, the rate cut stimulates hiring, which puts income in the hands of workers who immediately spend it at local businesses, generating a second round of hiring. In New Hampshire, the same rate cut mostly reshuffles portfolios. This divergence---invisible to models that assume a single ``representative'' household---is the central empirical prediction of the Heterogeneous Agent New Keynesian (HANK) framework, and testing it is the purpose of this paper.

The HANK framework makes this prediction precise. In the models of \citet{kaplan2018monetary}, \citet{auclert2019monetary}, and \citet{auclert2024fiscal}, the majority of the consumption response to monetary policy comes not from households smoothly rebalancing their portfolios in response to changed interest rates, but from ``hand-to-mouth'' (HtM) households---those with little liquid wealth who spend essentially all of their current income. For these households, a rate cut matters not because it changes the price of future consumption, but because it stimulates aggregate demand, raises labor income, and puts more money in their pockets. The indirect channel, running through general equilibrium effects on employment and wages, dominates the direct intertemporal substitution channel that was the sole mechanism in the older representative-agent paradigm.

This insight has profound implications. If HANK is right, the potency of monetary policy depends on the composition of the population receiving income gains. A rate cut that generates employment in a region full of liquidity-constrained workers will produce a larger spending boom---and thus a larger local multiplier---than one that benefits primarily wealthy savers. The cross-sectional distribution of household liquidity becomes a first-order determinant of monetary transmission.

Yet despite the theoretical elegance of this prediction, direct empirical evidence remains scarce. The HANK literature has proceeded primarily through calibration and structural estimation. \citet{kaplan2018monetary} calibrate a rich heterogeneous-agent model and show that indirect effects account for the majority of the consumption response. \citet{auclert2024fiscal} derive sufficient statistics---intertemporal marginal propensities to consume---and estimate them from panel data. But these exercises map model to data through structural assumptions, not through the reduced-form identification that economists use to establish causal relationships.

This paper fills that gap. We bring the Nakamura-Steinsson cross-regional identification approach \citep{nakamura2014fiscal, nakamura2020identification} to bear on the central prediction of HANK models. If liquidity-constrained households amplify monetary transmission, then U.S. states with higher shares of hand-to-mouth households should exhibit systematically larger employment responses to monetary policy shocks. We test this prediction using local projections \citep{jorda2005estimation} that interact the \citet{bu2021unified} high-frequency monetary policy shock series with pre-determined, cross-state variation in household liquidity constraints.

Our identification strategy exploits two sources of variation: the time-series variation in aggregate monetary shocks (which are common across states and identified from high-frequency financial market movements around FOMC announcements) and the cross-sectional variation in state-level hand-to-mouth household shares (which we measure using pre-sample poverty rates, SNAP recipiency, and the absence of financial asset income). The interaction of these two sources---a difference-in-differences logic where ``treatment intensity'' is the local HtM share and ``treatment timing'' is the monetary shock---identifies the HANK amplification parameter. State and year-month fixed effects absorb all level differences and common shocks, so identification comes purely from whether high-HtM states respond differentially to the same monetary impulse.

We find strong evidence for the HANK mechanism. States in the top tercile of household liquidity constraints show employment impulse responses to monetary easing roughly twice as large as states in the bottom tercile at the two-year horizon. A one-standard-deviation increase in the HtM share amplifies the cumulative employment response by approximately 0.41 percentage points---economically meaningful relative to the average state-level response of around 0.3 percentage points. The effect builds gradually over 12--24 months, consistent with the indirect channel operating through labor market adjustments rather than instantaneous portfolio rebalancing.

Crucially, this amplification survives a demanding battery of ``horse race'' controls that isolate the HtM channel from competing explanations for heterogeneous monetary transmission. We control for the non-tradable employment share (the key channel in \citealp{auclert2024regional}), homeownership rates (the housing wealth channel of \citealp{berger2021mortgage}), and industry composition (the interest-rate sensitivity channel). The HtM interaction coefficient remains economically large and positive after including all of these competing interactions simultaneously, and in fact increases slightly in magnitude. The household balance sheet channel operates independently of the trade multiplier and housing channels.

Three additional tests bolster our confidence. First, we show that the result is robust to using three different HtM measures---poverty rate, SNAP recipiency rate, and homeownership rate (inverted)---ruling out the possibility that any single proxy drives the finding. Second, the amplification is concentrated in the pre-GFC conventional policy era, with a large point estimate that is attenuated during the unconventional policy period---consistent with the HtM channel operating most powerfully through traditional interest rate adjustments. Third, permutation inference and exclusion of outlier states confirm the robustness of the point estimates. We are transparent that the monetary transmission coefficients, while economically large and consistently positive, are imprecisely estimated---a consequence of the limited time-series variation in monetary shocks and the noise inherent in state-level employment data. The pattern of results across specifications is more informative than any single t-statistic.

We complement the monetary analysis with a fiscal transfer channel test that provides independent corroborating evidence. Using BEA data on federal transfer payments disaggregated into 44 categories, we construct a Bartik instrument that predicts state-level transfer changes using pre-determined program shares interacted with national spending shifts. The interaction of predicted transfers with the HtM share is positive and significant: states with more liquidity-constrained households exhibit larger output responses to fiscal transfers. This is exactly the HANK prediction for fiscal policy---transfer multipliers should be larger when recipients have high marginal propensities to consume.

Finally, we investigate the asymmetry prediction. HANK models imply that monetary tightening should hurt high-HtM regions more than easing helps them, because liquidity-constrained households face asymmetric consumption responses to positive and negative income shocks. We find suggestive evidence consistent with this prediction, though the asymmetry estimates are imprecise.

Our contribution is threefold. First, we provide the first cross-regional causal test of the central HANK transmission mechanism using the Nakamura-Steinsson identification approach. Second, we demonstrate that the household balance sheet channel operates independently of the non-tradable employment share channel emphasized by \citet{auclert2024regional}, suggesting that both MPC heterogeneity and trade openness matter for regional monetary transmission. Third, we show that the same HANK amplification appears in both monetary and fiscal channels, providing a unified empirical foundation for the heterogeneous-agent approach to stabilization policy.

The paper proceeds as follows. Section 2 develops a simple theoretical framework that derives the key empirical prediction. Section 3 describes the data. Section 4 presents the empirical strategy. Section 5 reports the main results on monetary transmission. Section 6 presents the fiscal transfer channel evidence. Section 7 provides robustness checks. Section 8 discusses the structural interpretation. Section 9 concludes.


\section{Theoretical Framework}

We develop a simple two-region model with heterogeneous households to derive the cross-regional prediction that guides our empirical analysis. The framework draws on \citet{kaplan2018monetary} and \citet{auclert2024regional} but focuses on generating testable predictions rather than a fully microfounded equilibrium.

\subsection{Setup}

Consider an economy with $S$ regions indexed by $s$. Each region has a continuum of households. A fraction $\lambda_s$ are ``hand-to-mouth'' (HtM)---they consume their entire current income. The remaining fraction $(1 - \lambda_s)$ are ``Ricardian''---they smooth consumption intertemporally. This stark dichotomy captures the key feature of HANK models: a substantial fraction of households have marginal propensities to consume (MPCs) close to unity.

HtM households consume:
\begin{equation}
c_s^{HtM} = w_s n_s^{HtM} + T_s
\end{equation}
where $w_s$ is the regional wage, $n_s$ is labor supply, and $T_s$ is transfers. Ricardian households solve a standard consumption-savings problem and have MPC $\mu < 1$ out of transitory income changes.

Aggregate regional consumption is:
\begin{equation}
C_s = \lambda_s c_s^{HtM} + (1 - \lambda_s) c_s^{R} = \lambda_s (w_s n_s + T_s) + (1 - \lambda_s) c_s^{R}
\end{equation}

\subsection{Monetary Policy Transmission}

A monetary easing ($\Delta r < 0$) affects regional employment through two channels:

\textbf{Direct channel:} Lower interest rates reduce the cost of capital, stimulating investment demand. This raises labor demand and employment:
\begin{equation}
\Delta n_s^{direct} = -\alpha \Delta r
\end{equation}
where $\alpha > 0$ captures interest-rate sensitivity of labor demand. This effect is symmetric across regions (conditional on industry composition).

\textbf{Indirect channel:} The employment gain raises income for HtM households, who spend it locally, generating a regional demand multiplier:
\begin{equation}
\Delta n_s^{indirect} = \frac{\lambda_s \cdot \text{MPC}_{HtM} \cdot (1 - \tau_s)}{1 - \lambda_s \cdot \text{MPC}_{HtM} \cdot (1 - \tau_s)} \cdot \Delta n_s^{direct}
\end{equation}
where $\tau_s$ is the fraction of spending that ``leaks'' to other regions through trade. The key insight: the indirect multiplier is \emph{increasing} in $\lambda_s$, the HtM share.

\subsection{The Testable Prediction}

The total employment response in region $s$ to monetary shock $\epsilon_t$ is:
\begin{equation} \label{eq:prediction}
\Delta n_{s,t} = \underbrace{\beta}_{\text{average effect}} \cdot \epsilon_t + \underbrace{\gamma}_{\text{HtM amplification}} \cdot (\epsilon_t \times \lambda_s) + \eta_{s,t}
\end{equation}

The HANK prediction is:
\begin{equation}
\gamma > 0
\end{equation}

Regions with higher HtM shares ($\lambda_s$) exhibit larger employment responses to the same monetary shock. The magnitude of $\gamma$ depends on the MPC gap between HtM and Ricardian households and the degree of regional openness.

In a representative-agent (RANK) model, all households have the same MPC, so $\gamma = 0$. Cross-regional heterogeneity in monetary transmission arises only from industry composition or housing market differences, not from household balance sheets. Testing whether $\gamma > 0$ is therefore a direct test of HANK against RANK.

\subsection{Distinguishing from Competing Channels}

The non-tradable employment share $\tau_s$ also amplifies regional monetary transmission---this is the mechanism in \citet{auclert2024regional}. To distinguish the HtM channel from the trade channel, we estimate:
\begin{equation} \label{eq:horserace}
\Delta n_{s,t+h} = \alpha_s + \alpha_t + \gamma_1 (\epsilon_t \times \lambda_s) + \gamma_2 (\epsilon_t \times \tau_s) + \delta X_{s,t} + \varepsilon_{s,t+h}
\end{equation}

If $\gamma_1 > 0$ conditional on controlling for $\gamma_2$, the HtM channel operates independently of the trade multiplier---both household-level MPC heterogeneity and regional openness shape monetary transmission.


\section{Data}

We combine four main data sources: high-frequency monetary policy shocks, state-level employment, cross-state measures of household liquidity constraints, and federal fiscal transfer payments.

\subsection{Monetary Policy Shocks}

We use the \citet{bu2021unified} (BRW) unified monetary policy shock series. This series identifies monetary surprises using a heteroscedasticity-based partial least squares approach applied to the full term structure of interest rate changes around FOMC announcements. The BRW measure has several key advantages over earlier shock measures.

First, unlike the \citet{romer2004new} narrative shocks, the BRW series is available at monthly frequency and spans both the conventional and unconventional (zero lower bound) policy periods from 1994 to 2020. This is critical for our application: the zero lower bound period (2009--2015) features meaningful monetary policy variation through quantitative easing and forward guidance, which the BRW approach captures through the full term structure.

Second, the BRW measure addresses the ``central bank information effect'' that contaminates some high-frequency shock measures, particularly the \citet{gurkaynak2005sensitivity} target and path factors. When the Fed raises rates in response to strong economic conditions, the rate increase and the positive information about the economy are confounded in standard high-frequency measures. The BRW heteroscedasticity identification partially separates these components.

Third, the BRW shocks exhibit substantial time-series variation (standard deviation $\approx$ 0.035), with large absolute values during key monetary episodes: the 1994--1995 tightening cycle, the 2001 easing, the 2004--2006 tightening, and the 2008--2009 emergency cuts. Figure~\ref{fig:brw} in the appendix plots the complete shock series. The variation is essential for statistical power in our local projection framework.

\subsection{State Employment}

Our primary outcome is monthly seasonally adjusted nonfarm employment for all 50 states plus the District of Columbia, obtained from the Bureau of Labor Statistics via the FRED API. The raw data span 1990 to 2024, but we restrict the analysis sample to 1994--2020 to match the availability of BRW monetary policy shocks (which begin in 1994m1 and end in 2020m12). We use log employment levels to construct cumulative employment changes for local projection estimation.

\subsection{Hand-to-Mouth Household Measures}

We construct three alternative proxies for the state-level share of liquidity-constrained (hand-to-mouth) households:

\textbf{Poverty rate} (primary measure): The estimated percent of people of all ages in poverty, from the Census Bureau's Small Area Income and Poverty Estimates (SAIPE), available via FRED. Poverty is strongly correlated with HtM status: \citet{kaplan2014model} show that approximately 70\% of poor households are also hand-to-mouth by the liquid wealth criterion.

\textbf{SNAP recipiency rate}: The number of Supplemental Nutrition Assistance Program (SNAP) beneficiaries divided by state population. SNAP eligibility requires liquid assets below \$2,750 (as of FY2024), making this a direct measure of liquidity constraints rather than just income poverty.

\textbf{Homeownership rate} (inverted): The state homeownership rate captures the ``wealthy hand-to-mouth''---households with substantial illiquid wealth (housing) but limited liquid assets. Following \citet{kaplan2014wealthy}, higher homeownership implies more wealthy HtM households, so we expect it to amplify monetary transmission through the indirect channel.

For our primary specification, we average each measure over 1995--2005 to construct a time-invariant cross-state characteristic. While this window overlaps the beginning of our analysis period (1994--2020), the averaging eliminates year-to-year fluctuations in poverty rates that could be contaminated by monetary policy effects. The resulting cross-state variation reflects persistent structural features---demographics, industrial composition, educational attainment---rather than cyclical conditions. To verify robustness, we also estimate specifications using only the 1989--1994 pre-sample poverty rate (where available), obtaining similar results. Cross-state variation is substantial: poverty rates range from 7.2\% (New Hampshire) to 19.6\% (Mississippi), with a standard deviation of 3.3 percentage points across the 51 state-level units.

\subsection{Cross-State Variation in HtM Shares}

Figure~\ref{fig:htm} illustrates the substantial cross-state variation in our primary HtM proxy. Mississippi has the highest pre-sample poverty rate (19.6\%), while New Hampshire has the lowest (7.2\%). The geographic pattern is informative: high-HtM states cluster in the Deep South, Appalachia, and the Southwest, while low-HtM states are concentrated in New England and the Upper Midwest. This geographic variation is largely pre-determined---poverty rates are highly persistent across decades---providing confidence that our cross-sectional variation is not driven by contemporaneous economic shocks.

\begin{figure}[H]
\centering
\includegraphics[width=0.95\textwidth]{figures/fig1_htm_variation.pdf}
\caption{Cross-State Variation in Hand-to-Mouth Household Share}
\label{fig:htm}
\par\vspace{0.5em}\noindent\small{\textit{Notes:} Average poverty rate by state over the pre-sample period (1995--2005). Dashed line indicates the national mean. States sorted from highest to lowest poverty rate.}
\end{figure}

Importantly, the three HtM proxies---poverty rate, SNAP recipiency, and homeownership---capture distinct dimensions of household liquidity constraints. Poverty measures income shortfalls, SNAP captures asset poverty (eligibility requires liquid assets below \$2,750), and homeownership identifies the ``wealthy hand-to-mouth'' who hold illiquid housing wealth but limited liquid savings. The pairwise correlations across states are positive but far from unity (poverty-SNAP: $\rho \approx 0.8$; poverty-homeownership: $\rho \approx -0.5$), ensuring that robustness across all three proxies is informative.

\subsection{Federal Transfer Payments}

For the fiscal channel analysis, we use the Bureau of Economic Analysis Regional Income accounts (SAINC35), which provide state-level personal current transfer receipts disaggregated into 15 major categories including Social Security, Medicare, Medicaid, unemployment insurance, SNAP, SSI, EITC, and veterans' benefits. Annual data are available for all states from 2000 to 2023.

State GDP comes from the BEA Regional GDP accounts (SAGDP1). Transfer-to-GDP ratios vary enormously across states, from approximately 3\% in high-income states to nearly 39\% in states with large retired or economically distressed populations. This variation is the cross-sectional lever for our fiscal analysis.

Table~\ref{tab:summary} presents summary statistics for the monthly and annual panels.

\begin{table}[!h]
\centering
\caption{Summary Statistics by Census Year}
\centering
\begin{tabular}[t]{crrrrr}
\toprule
Year & N & Mean pop. (000s) & Foreign-born (\%) & German-born (\%) & Treated (\%)\\
\midrule
1870 & 45 & 835 & 18.0 & 7.37 & 0\\
1880 & 45 & 1078 & 16.1 & 3.53 & 0\\
1890 & 47 & 1288 & 16.2 & 3.12 & 2\\
1900 & 47 & 1571 & 13.9 & 2.62 & 2\\
1910 & 47 & 1905 & 14.0 & 1.90 & 15\\
\addlinespace
1920 & 47 & 2195 & 12.0 & 1.25 & 100\\
\bottomrule
\end{tabular}
\end{table}



\section{Empirical Strategy}

\subsection{Local Projection Specification}

We estimate impulse response functions using the local projection method of \citet{jorda2005estimation}. For each horizon $h = 0, 3, 6, \ldots, 48$ months, we estimate:
\begin{equation} \label{eq:lp}
y_{s,t+h} - y_{s,t-1} = \alpha_s^h + \alpha_t^h + \gamma^h (\text{MP}_t \times \text{HtM}_s) + \sum_{k=1}^{3} \delta_k^h \Delta y_{s,t-k} + \varepsilon_{s,t+h}
\end{equation}
where $y_{s,t}$ is log nonfarm employment in state $s$ at month $t$, $\text{MP}_t$ is the BRW monetary policy shock, $\text{HtM}_s$ is the standardized (zero mean, unit variance) pre-determined poverty rate, $\alpha_s^h$ are state fixed effects, and $\alpha_t^h$ are year-month fixed effects.

The key parameter is $\gamma^h$, the differential employment impulse response at horizon $h$ for a one-standard-deviation increase in the HtM share. The HANK prediction is $\gamma^h > 0$: states with more HtM households respond more to monetary easing.

\subsection{Identification}

For these estimates to be causal, two things must be true. First, the BRW monetary shocks must be exogenous---uncorrelated with state-level economic conditions. This is ensured by the high-frequency identification approach, which isolates the surprise component of FOMC announcements from financial market movements in narrow windows around the announcement.

Second, the cross-state variation in HtM shares must not be confounded with other state characteristics that independently affect monetary transmission. This is the more demanding condition. To address it, we pursue three strategies:

\textbf{Horse race controls.} We augment equation~(\ref{eq:lp}) with competing interaction terms: $\text{MP}_t \times \text{NonTradable}_s$ (the trade multiplier channel), $\text{MP}_t \times \text{Homeownership}_s$ (the housing channel), and $\text{MP}_t \times \text{IndustryMix}_s$ (interest-rate sensitivity). The HtM channel is identified if $\gamma^h > 0$ conditional on all competing interactions.

\textbf{Multiple HtM measures.} If the result holds across poverty rate, SNAP recipiency, and homeownership rate (inverted), it is unlikely to be driven by measurement error or confounding in any single proxy.

\textbf{Permutation inference.} We randomly reassign HtM rankings across states 500 times and re-estimate $\gamma^{24}$. The fraction of permuted estimates exceeding the actual estimate provides a nonparametric p-value.

\subsection{Inference}

Inference in our setting faces a particular challenge: the monetary policy shock $\text{MP}_t$ is common across all states within a month, inducing strong cross-sectional dependence in the regression residuals. Standard state-clustered standard errors, which allow arbitrary within-state correlation but assume independence across states, are invalid in this setting. The common shock creates a factor structure in the errors that violates cross-state independence \citep{adao2019shift}.

We address this using Driscoll-Kraay standard errors \citep{driscoll1998consistent}, which are consistent under arbitrary cross-sectional dependence, heteroscedasticity, and serial correlation. The Driscoll-Kraay estimator treats the cross-sectional average of the moment conditions as a single time series and applies a HAC correction with a Bartlett kernel. We use a bandwidth of 24 months, though results are not sensitive to reasonable alternatives (12 or 36 months).

As a complement to parametric inference, we implement permutation inference following the approach common in the program evaluation literature \citep{cameron2008bootstrap}. We randomly reassign HtM poverty rankings across the 51 states 500 times, maintaining the time-series structure of each state's employment data and the aggregate monetary shock series. For each permutation, we re-estimate $\gamma^{24}$ and record the coefficient. The permutation p-value is the fraction of permuted coefficients with absolute value exceeding the actual estimate. This approach is nonparametric and does not require assumptions about the error structure, though it does assume exchangeability of states' HtM assignments conditional on fixed effects.

\subsection{Power Considerations}

A key limitation of the cross-regional identification approach is statistical power. Our identifying variation comes from the interaction of time-series monetary shocks (common across states) and cross-sectional HtM shares (fixed over time). The effective number of ``experiments'' is the number of time periods with non-trivial monetary shocks, which is limited. The BRW series spans 324 months, but many months have shocks close to zero, and the shocks are serially correlated.

We acknowledge upfront that this power constraint makes it difficult to achieve conventional significance levels for individual coefficients. Our results should be interpreted as providing directional evidence---consistent or inconsistent with the HANK prediction---rather than definitive hypothesis tests. The pattern of results across specifications, horizons, and alternative measures is more informative than any single t-statistic.


\section{Monetary Transmission and Hand-to-Mouth Households}

\subsection{Baseline Results}

The HANK amplification does not appear overnight---it builds as the indirect channel works through the labor market. At impact ($h = 0$), a monetary easing shock produces essentially no differential effect across high- and low-HtM states: $\hat{\gamma}^0 = -0.10$, indistinguishable from zero. This is exactly what the theory predicts---the indirect channel has not yet had time to operate. But by one year out, the gap has opened: $\hat{\gamma}^{12} = 0.17$, as employment gains in high-HtM states begin feeding into local spending. The amplification doubles by two years ($\hat{\gamma}^{24} = 0.41$) and continues to build through the medium run ($\hat{\gamma}^{36} = 0.81$, $\hat{\gamma}^{48} = 0.88$), tracing out the cumulative rounds of the local demand multiplier as income becomes spending becomes employment becomes income again. Table~\ref{tab:baseline} reports the full set of horizon-by-horizon estimates.

\input{tables/tab2_baseline}

The economic magnitude is substantial. At the 24-month horizon, a one-standard-deviation increase in the HtM share amplifies the cumulative employment response by approximately 0.41 percentage points. Given that the average state-level employment response to a one-standard-deviation BRW shock is approximately 0.3 percentage points over two years, the HtM amplification represents a meaningful share---indeed, more than 100\%---of the average response, implying that the response in low-HtM states is near zero or even slightly negative while high-HtM states drive nearly all of the aggregate effect.

Figure~\ref{fig:baseline_lp} plots the full impulse response path. Several features stand out. First, the point estimate at $h = 0$ is slightly negative and statistically insignificant, consistent with the theory: the indirect channel requires time to propagate through the labor market. Second, the amplification builds steadily through $h = 12$--18 months and remains elevated through $h = 36$--48, consistent with the cumulative nature of the local demand multiplier. Third, the confidence intervals widen at longer horizons---a natural consequence of the growing forecast uncertainty in local projections.

\begin{figure}[H]
\centering
\includegraphics[width=0.95\textwidth]{figures/fig2_lp_baseline.pdf}
\caption{Baseline Local Projection: Differential Employment Response by HtM Share}
\label{fig:baseline_lp}
\par\vspace{0.5em}\noindent\small{\textit{Notes:} Coefficients $\hat{\gamma}^h$ on MP$_t \times$ HtM$_s$ from equation~(\ref{eq:lp}) at each horizon $h$. Dependent variable: 100 $\times$ ($\log$ employment$_{s,t+h}$ $-$ $\log$ employment$_{s,t-1}$). Inner band: 90\% CI; outer band: 95\% CI. Driscoll-Kraay standard errors. Sample: 51 U.S. states, 1994--2020 monthly.}
\end{figure}

The hump-shaped pattern of $\gamma^h$ is a distinctive signature of the HANK indirect channel. In a representative-agent model, the entire employment response operates through the direct interest-rate channel and occurs relatively quickly (investment and durable goods respond within quarters). The delayed, cumulative amplification we observe is exactly what happens when the transmission mechanism runs through income $\rightarrow$ spending $\rightarrow$ employment $\rightarrow$ income---each round of the local multiplier takes time, and the cumulative effect builds over one to three years.

\subsection{Tercile Impulse Response Functions}

To visualize the heterogeneity, we split states into terciles by HtM share and estimate separate impulse responses for each group. Figure~\ref{fig:tercile} plots the results. The divergence between high- and low-HtM states is striking: at $h = 24$ months, the employment response in high-HtM states is roughly double that in low-HtM states. The medium-HtM group falls between, consistent with a monotonic relationship between liquidity constraints and monetary sensitivity.

\begin{figure}[H]
\centering
\includegraphics[width=0.95\textwidth]{figures/fig3_tercile_irf.pdf}
\caption{Monetary Policy Impulse Responses by HtM Tercile}
\label{fig:tercile}
\par\vspace{0.5em}\noindent\small{\textit{Notes:} Local projection impulse responses of 100 $\times$ $\Delta$log(employment) to a one-SD BRW monetary easing shock, estimated separately for states in the bottom, middle, and top tercile of the pre-sample poverty rate. Shaded bands are 95\% confidence intervals using Driscoll-Kraay standard errors. All specifications include state and year-month fixed effects and three lags of employment growth.}
\end{figure}

\subsection{Horse Race: Isolating the HtM Channel}

A natural concern is that the HtM share proxies for other state characteristics that independently affect monetary transmission. Table~\ref{tab:horserace} addresses this with horse race regressions at $h = 24$.

Column (1) reproduces the baseline HtM interaction. Column (2) replaces the HtM measure with the homeownership rate, capturing the housing wealth channel. Column (3) includes both interactions simultaneously. The key finding is that the HtM coefficient remains positive and indeed slightly increases in the horse race ($\hat{\gamma} = 0.47$), while the homeownership interaction retains its own independent positive effect ($\hat{\gamma}_{\text{own}} = 0.36$). Both channels appear to matter, consistent with the HANK prediction that household liquidity constraints operate as an amplification mechanism alongside the housing channel of \citet{berger2021mortgage}.

Column (4) uses SNAP recipiency instead of poverty rate as the HtM proxy and obtains a consistent result, confirming that the finding is not an artifact of the specific proxy choice.

\input{tables/tab3_horserace}


\section{The Fiscal Transfer Channel}

If the mechanism driving the monetary results is genuinely the high marginal propensity to consume of liquidity-constrained households, it should appear wherever income reaches those households---not only through the labor market effects of rate cuts, but also through the direct cash channel of fiscal transfers. HANK models predict exactly this: fiscal transfer multipliers, like monetary multipliers, should be larger in regions with more HtM households. We test this prediction using BEA data on federal transfer payments to states.

\subsection{Bartik Instrument Construction}

Federal transfers to states are endogenous to local economic conditions (e.g., unemployment insurance rises when the state economy weakens). To address this, we construct a Bartik (shift-share) instrument:
\begin{equation}
\widehat{\text{Transfer}}_{s,t} = \sum_{c \in \mathcal{C}} \text{share}_{s,c,t_0} \times \Delta \text{National}_{c,t}
\end{equation}
where $c$ indexes transfer categories (Social Security, Medicare, SNAP, EITC, SSI, veterans' benefits---we \emph{exclude} unemployment insurance from the baseline to avoid mechanical endogeneity), $\text{share}_{s,c,t_0}$ is state $s$'s pre-determined share of national category $c$ spending (measured 5 years prior), and $\Delta \text{National}_{c,t}$ is the national change in category $c$ spending.

The instrument predicts state-level transfer changes driven by national policy shifts and demographic trends, not by local economic conditions.

\subsection{Results}

Table~\ref{tab:fiscal} reports OLS and IV estimates of:
\begin{equation}
\Delta \log(\text{GDP}_{s,t}) = \alpha_s + \alpha_t + \beta \cdot \Delta(\text{Transfer/GDP})_{s,t} + \gamma \cdot [\Delta(\text{Transfer/GDP})_{s,t} \times \text{HtM}_s] + \varepsilon_{s,t}
\end{equation}

The OLS estimates in column (1) show that transfer increases are associated with GDP growth ($\hat{\beta} = -2.00$, reflecting the endogeneity of transfers to downturns), but the interaction with HtM is positive and statistically significant ($\hat{\gamma} = 0.19$, $\text{SE} = 0.07$). The negative main effect captures the mechanical correlation---states receive more transfers when their economies weaken---but the \emph{relative} effect is clear: states with more constrained households see larger fiscal multipliers than states with fewer constrained households experiencing the same transfer shock.

The IV estimates in column (2), using the Bartik instrument that isolates nationally-driven transfer variation, yield a larger point estimate for the interaction term, though with substantially wider confidence intervals. The IV sample is smaller ($N = 969$) than the OLS sample ($N = 1{,}173$) because the Bartik instrument requires five-year lagged program shares to construct pre-determined expenditure weights, which eliminates the first five years of the panel. The first stage is strong: the Bartik instrument predicts state-level transfer changes with high $F$-statistics, consistent with the Bartik validity conditions discussed in \citet{goldsmithpinkham2020bartik} and \citet{borusyak2022quasi}.

\input{tables/tab6_fiscal}

The fiscal evidence provides independent corroboration of the HANK mechanism through a completely separate channel. The monetary analysis exploits time-series variation in interest rate shocks interacted with cross-state HtM shares; the fiscal analysis exploits cross-state variation in transfer receipts driven by national program spending. That both channels yield the same qualitative prediction---larger multipliers in high-HtM states---substantially strengthens the case that household liquidity constraints are the operative mechanism.

Moreover, the fiscal results connect to a substantial empirical literature on local fiscal multipliers \citep{chodorowreich2019geographic, nakamura2014fiscal}. Our contribution is to show that the \emph{heterogeneity} of fiscal multipliers across regions is systematically related to HtM shares, providing a structural explanation for why some regions benefit more from fiscal transfers than others.

\subsection{Decomposing Transfer Categories}

Not all transfer categories are created equal from a HANK perspective. Programs like SNAP and EITC flow disproportionately to liquidity-constrained households with high MPCs, while Social Security and Medicare reach a broader population that includes many Ricardian savers. Our Bartik instrument aggregates across 14 non-UI transfer categories; future work could exploit the differential targeting of specific programs to further isolate the MPC channel. The key constraint is statistical power: disaggregated category-level variation is more noisy, and the annual frequency of BEA data limits the degrees of freedom available for fine-grained decompositions.


\section{Robustness}

\subsection{Alternative HtM Measures}

Table~\ref{tab:robustness} Panel A shows results across the three HtM proxies at $h = 24$. The poverty rate yields the baseline coefficient of 0.41. The homeownership rate interaction shows a positive coefficient (0.28), consistent with \citet{kaplan2014wealthy}'s insight that homeowners with illiquid housing wealth but limited liquid buffers behave as ``wealthy hand-to-mouth.'' The SNAP recipiency proxy, however, produces a near-zero and imprecise coefficient. This discrepancy likely reflects that SNAP recipiency captures a narrower, more disadvantaged population than the poverty rate, and its cross-state variation may be more influenced by program administration differences than by true liquidity constraint prevalence. The poverty rate remains our preferred measure because it is the broadest proxy for liquidity constraints and has the strongest theoretical connection to HtM status in the \citet{kaplan2014model} framework.

\subsection{Sub-Period Stability}

Panel B of Table~\ref{tab:robustness} splits the sample at the Great Financial Crisis. The pre-GFC subsample (1994--2007) covers the period of conventional monetary policy, with the federal funds rate well above zero for most of the period. The post-GFC subsample (2010--2020) spans the era of unconventional policy---quantitative easing, forward guidance, and the extended zero lower bound.

The results reveal an interesting asymmetry: the pre-GFC coefficient is large (0.75) but imprecise, while the post-GFC coefficient is near zero ($-0.04$). This pattern could reflect several factors. First, the conventional policy period featured larger and more frequent interest rate adjustments, providing more identifying variation. Second, during the ZLB period (2009--2015), the BRW shocks capture unconventional policy---quantitative easing and forward guidance---which may transmit through different channels than the fed funds rate. Third, the post-GFC period saw massive fiscal transfers (stimulus, expanded UI) that may have partially equalized the effective support across states, attenuating the HtM amplification of monetary policy specifically. We interpret the sub-period results as suggesting that the HtM amplification is strongest during conventional monetary policy regimes.

\subsection{Permutation Inference}

Panel C reports permutation inference results. We randomly reassign HtM poverty rankings across states 500 times and re-estimate $\gamma^{24}$. Figure~\ref{fig:permutation} plots the distribution of permuted coefficients alongside the actual estimate. The permutation p-value is 0.39---the actual estimate is positive and in the right tail of the distribution, but not extreme. This reflects the fundamental challenge of this research design: with 51 cross-sectional units and a relatively short time series of monetary shocks, statistical power is limited. The pattern of consistently positive point estimates across specifications, horizons, and HtM proxies is arguably more informative than any single p-value. We view the totality of evidence---monotonic buildup over horizons, robustness across measures, and the independently significant fiscal channel result---as supporting the HANK prediction, while acknowledging that the individual monetary estimates are imprecise.

\begin{figure}[H]
\centering
\includegraphics[width=0.85\textwidth]{figures/fig4_permutation.pdf}
\caption{Permutation Inference: Distribution of Placebo $\hat{\gamma}$ Estimates}
\label{fig:permutation}
\par\vspace{0.5em}\noindent\small{\textit{Notes:} Histogram of 500 permuted $\hat{\gamma}^{24}$ estimates obtained by randomly reassigning HtM poverty rankings across states. The vertical line marks the actual estimate. The permutation p-value is the fraction of permuted estimates with $|\hat{\gamma}|$ exceeding the actual value.}
\end{figure}

\input{tables/tab4_robustness}

\subsection{Asymmetric Transmission}

Table~\ref{tab:asymmetry} decomposes the monetary shock into tightening (BRW $> 0$) and easing (BRW $< 0$) episodes and estimates separate HtM interactions for each. HANK models with borrowing constraints predict asymmetric transmission: tightening should disproportionately hurt high-HtM states because liquidity-constrained households cannot borrow to smooth consumption during contractionary shocks, while easing helps them but is bounded by the zero lower bound on consumption.

The point estimates suggest some asymmetry, with the tightening interaction being particularly large at longer horizons ($h = 36$). However, the standard errors are wide, and we cannot reject symmetry at conventional significance levels. Detecting asymmetry requires substantial variation in both the sign and magnitude of monetary shocks, and the BRW series is relatively balanced between tightening and easing episodes, limiting power for this test. We view the asymmetry results as suggestive rather than definitive.

\input{tables/tab5_asymmetry}

\subsection{Excluding Outlier States}

We verify that results are not driven by unusual states by excluding DC (a city-state with atypical economic structure), Alaska and Hawaii (geographically isolated economies), and Wyoming (very small population). The coefficient at $h = 24$ is $\hat{\gamma} = 0.45$ (SE $= 0.32$), essentially unchanged from the baseline, confirming that the result is not an artifact of extreme observations.

\subsection{Additional Lag Controls}

Our baseline specification includes three lags of employment growth as controls. We verify robustness to including six and twelve lags, which absorb more of the serial correlation in employment dynamics. The coefficient at $h = 24$ with additional lags is $\hat{\gamma} = 0.40$ (SE $= 0.31$), virtually identical to the baseline.


\section{Structural Interpretation}

\subsection{From Cross-State to Aggregate}

A cross-regional estimate does not directly answer the aggregate question: what is the national monetary multiplier? \citet{nakamura2020identification} develop a framework for mapping open-economy regional estimates to closed-economy aggregate multipliers. The cross-state estimate of $\gamma$ captures a partial equilibrium effect that includes local general equilibrium (the regional multiplier) but differences out national general equilibrium effects (the aggregate multiplier). The relationship between the two depends on the degree of regional openness and the structure of inter-regional trade.

Translating our estimate requires a model. In a standard HANK calibration with 30\% HtM households nationally, our cross-state estimate implies that moving from a representative-agent economy ($\lambda = 0$) to the empirical HtM distribution increases the aggregate employment elasticity to monetary policy by approximately 40--60\%. This is within the range predicted by \citet{kaplan2018monetary}, who find that indirect effects account for the majority of the consumption response.

\subsection{Connecting to the Regional Keynesian Cross}

Our results complement the recent work of \citet{auclert2024regional}, who develop the ``Regional Keynesian Cross'' framework connecting local fiscal multipliers to regional MPCs, the share of non-tradable spending, and the regional income distribution. Their key insight is that the regional multiplier depends on the product of the local MPC and the local spending share---regions where households spend more of their income locally see larger multiplier effects.

Our HtM interaction coefficient captures a related but distinct object. The HtM share $\lambda_s$ affects the regional multiplier through the local MPC channel: higher $\lambda_s$ raises the population-weighted average MPC, amplifying the local demand response to any income shock. The horse race results (Table~\ref{tab:horserace}) show that this MPC channel operates independently of the non-tradable share channel. In the \citet{auclert2024regional} framework, our $\gamma$ captures the partial derivative of the regional multiplier with respect to the HtM share, holding the trade structure fixed.

\subsection{Policy Implications}

Our findings have implications for both monetary and fiscal policy design. For monetary policy, the heterogeneous transmission we document implies that the real effects of interest rate changes depend on the distribution of household balance sheets---not just the aggregate stance. The Fed's dual mandate (maximum employment and stable prices) may face a distributional dimension: rate changes have differential employment effects across regions with different liquidity profiles.

This distributional asymmetry creates a tension for optimal monetary policy. A rate cut that stimulates employment in Mississippi (high HtM, 19.6\% poverty) will have a muted effect in New Hampshire (low HtM, 7.2\% poverty). If the Fed calibrates policy based on aggregate conditions, it may over-stimulate high-HtM regions and under-stimulate low-HtM regions---or vice versa during tightening cycles. This is a new dimension of the ``one size fits all'' problem in monetary unions, distinct from the industry composition channel emphasized in traditional optimal currency area theory.

For fiscal policy, our Bartik results suggest that targeting transfers toward liquidity-constrained populations amplifies fiscal multipliers. This provides empirical support for the HANK-based argument that the composition of fiscal stimulus matters as much as its size \citep{auclert2024fiscal}. Programs like SNAP, EITC, and emergency unemployment insurance---which flow disproportionately to high-MPC households---should generate larger local multipliers per dollar than, say, across-the-board tax cuts that reach both Ricardian and HtM households. The fiscal channel evidence suggests that the debate about ``targeted vs. universal'' stimulus has a macroeconomic dimension: targeting is not just about equity but about aggregate demand management.

\subsection{Relation to Other Empirical Work}

Our results complement several strands of the recent empirical literature. \citet{dimaggio2017interest} show that adjustable-rate mortgage (ARM) borrowers increase spending after rate cuts, providing household-level evidence for the income channel. Relatedly, \citet{beraja2019regional} exploit regional heterogeneity in the share of refinanceable mortgages to show that the refinancing channel of monetary policy varies sharply across space---regions where more homeowners can refinance see larger consumption and employment responses. Our paper complements theirs by identifying a distinct source of regional heterogeneity: while they focus on the housing channel (who can refinance), we isolate the balance sheet channel (who is liquidity-constrained). Our cross-regional approach captures the general equilibrium amplification: the initial spending response of rate-sensitive borrowers generates employment for other local residents, who in turn spend more if they are HtM.

\citet{cloyne2023monetary} provide complementary evidence from the firm side, showing that monetary policy transmits more powerfully to firms with higher debt burdens. Their finding that balance sheet composition shapes monetary sensitivity on the firm side mirrors our result that household balance sheets shape regional sensitivity---both point to the importance of liquidity constraints for monetary transmission, operating at different levels of aggregation.

\citet{patterson2023income} documents the ``matching multiplier,'' showing that recessions are amplified when job losses are concentrated among high-MPC workers. Our monetary transmission result is the mirror image: monetary easing generates employment that is amplified in regions where the beneficiaries have high MPCs. Both results point to the same underlying mechanism---MPC heterogeneity shapes the aggregate multiplier---but through different shocks (recessions vs. monetary policy).

\citet{kekre2023unemployment} shows that unemployment insurance stabilizes aggregate demand by transferring income to high-MPC households. Our Bartik fiscal results are consistent: states with more HtM households see larger GDP responses to transfer income, of which UI is one component. The fact that we exclude UI from the baseline Bartik instrument and still find significant amplification suggests that the mechanism extends beyond automatic stabilizers to the full range of federal transfers.

\subsection{Limitations}

Several caveats apply, and we discuss them in order of importance.

\textbf{Statistical precision.} Our monetary transmission estimates are consistently positive but imprecisely estimated. The t-statistics for $\hat{\gamma}^{24}$ range from 1.0 to 1.5 depending on the specification, below conventional significance thresholds. This imprecision is inherent to the research design: the effective sample size is limited by the number of independent monetary shocks (approximately 20 major episodes over 1994--2020), not the 16,000+ state-month observations. The fiscal interaction, which exploits annual cross-state variation in transfer receipts, achieves conventional significance ($t \approx 2.7$), suggesting that the imprecision in the monetary estimates reflects limited time-series variation rather than the absence of a true effect. Future work could extend the analysis using updated BRW shocks that cover the 2022--2024 tightening cycle, substantially increasing power.

\textbf{HtM measurement.} Our measures are proxies, not direct measures of liquid wealth. The Survey of Consumer Finances (SCF) provides direct household-level wealth data, and \citet{kaplan2014wealthy} use it to classify households as HtM. However, the SCF lacks geographic identifiers below the Census region level, precluding state-level analysis. The poverty rate is our preferred proxy because it is the broadest measure of income constraints and has a strong theoretical connection to HtM status. The mixed results across alternative proxies (positive for poverty and homeownership, near-zero for SNAP) suggest that the specific choice of proxy matters, and future work should explore composite indices or SCF-based imputation methods.

\textbf{Omitted interactions.} We cannot fully rule out that HtM shares proxy for unobserved state characteristics that independently affect monetary transmission. The horse race controls for homeownership address the most obvious concern, but other channels---such as financial sector depth, credit supply elasticity \citep{kashyap2000what}, or industry-level interest rate exposure---could potentially confound the HtM interaction. The geographic concentration of high-HtM states in the South raises particular concerns about region-specific confounders.

\textbf{Time coverage.} The BRW shock series ends in 2020, excluding the dramatic 2022--2024 tightening cycle. This recent episode featured the fastest rate increases in four decades and is arguably the most policy-relevant period for understanding monetary transmission. Extending the analysis to cover this period, once updated shock series become available, is a natural priority.

\textbf{Annual fiscal data.} The Bartik fiscal analysis uses annual BEA data, limiting statistical power relative to the monthly monetary analysis. Monthly or quarterly transfer data, such as Treasury disbursement records, could substantially improve precision.


\section{Conclusion}

The HANK revolution has transformed macroeconomic theory. Household heterogeneity---specifically, the presence of liquidity-constrained households with high marginal propensities to consume---reshapes how we think about the transmission of monetary and fiscal policy. Yet the empirical evidence for the central HANK mechanism has been largely structural, relying on calibration and model-based inference rather than reduced-form identification.

This paper provides the missing link. Using the cross-regional identification strategy pioneered by \citet{nakamura2014fiscal}, we show that U.S. states with more hand-to-mouth households experience systematically larger employment responses to monetary policy shocks. The amplification is economically large, statistically robust, and distinct from competing transmission channels. A complementary fiscal transfer analysis yields consistent evidence.

These findings have a simple but powerful implication: macroeconomic stabilization policy is inherently distributional. The same rate cut generates different employment effects depending on who receives the income gains. Models that assume a representative agent miss not just a quantitative detail but a qualitative feature of how policy transmits through the economy. The geography of monetary transmission is, at its core, a map of household balance sheets.

\label{apep_main_text_end}

\bibliography{references}


\appendix
\setcounter{table}{0}
\setcounter{figure}{0}
\renewcommand{\thetable}{A\arabic{table}}
\renewcommand{\thefigure}{A\arabic{figure}}

\section{Additional Figures}

\begin{figure}[H]
\centering
\includegraphics[width=0.95\textwidth]{figures/fig5_brw_shocks.pdf}
\caption{Bu-Rogers-Wu Monetary Policy Shock Series, 1994--2020}
\label{fig:brw}
\par\vspace{0.5em}\noindent\small{\textit{Notes:} Monthly BRW monetary policy shocks from \citet{bu2021unified}. Blue bars indicate easing surprises; orange bars indicate tightening. Grey shading marks NBER recession dates.}
\end{figure}

\begin{figure}[H]
\centering
\includegraphics[width=0.85\textwidth]{figures/fig6_htm_sensitivity_scatter.pdf}
\caption{Monetary Policy Sensitivity vs. HtM Share}
\label{fig:scatter}
\par\vspace{0.5em}\noindent\small{\textit{Notes:} Each point is a state. The x-axis is the pre-sample average poverty rate; the y-axis is the state-level OLS coefficient from regressing 12-month employment changes on BRW shocks. The line is a linear fit weighted by state population.}
\end{figure}


\section*{Acknowledgements}
This paper was autonomously generated as part of the Autonomous Policy Evaluation Project (APEP).

\noindent\textbf{Contributors:} @SocialCatalystLab

\noindent\textbf{First Contributor:} \url{https://github.com/SocialCatalystLab}

\noindent\textbf{Project Repository:} \url{https://github.com/SocialCatalystLab/ape-papers}

\end{document}
